% When using TeXShop on the Mac, let it know the root document. The following must be one of the first 20 lines.
% !TEX root = ../design.tex

\chapter[Graph]{Graph}

\begin{moduleinfo}
\item[Author] \href{mailto:okislal@pivotal.io}{Orhan Kislal}
\item[History]
	\begin{modulehistory}
		\item[v0.1] Initial version, SSSP only.
	\end{modulehistory}
\end{moduleinfo}


% Abstract. What is the problem we want to solve?

This module implements various graph algorithms that are used in a number of applications such as social networks, telecommunications and road networks.

% \section{Graph Representation} \label{sec:graph:rep}

% Our graph representation depends on two structures, a \emph{vertex} table and an \emph{edge} table.

\section{Single Source Shortest Path} \label{sec:graph:sssp}

Given a graph and a source vertex, single source shortest path (SSSP) algorithm finds a path for every vertex such that the sum of the weights of its constituent edges is minimized.

Shortest path is defined as follows. Let $e_{i,j}$ be the edge from vertex $i$ to vertex $j$ and $w_{i,j}$ be its weight. Given a graph G, the shortest path from $s$ to $d$ is $P = (v_1, v_2 \dots, v_n)$ (where $v_1=s$ and $v_n=d$) that over all possible $n$ minimizes the sum $ \sum _{i=1}^{n-1}f(e_{i,i+1})$.

% \subsection{Bellman Ford Algorithm}

Bellman-Ford Algorithm \cite{bellman1958routing,ford1956network} is based on the following idea: We start with a naive approximation for the cost of reaching every vertex. At each iteration, these values are refined based on the edge list and the existing approximations. If there are no refinements at any given step, the algorithm returns the calculated results. If the algorithm does not converge in $|V|-1$ iterations, this indicates the existence of a negative cycle in the graph.


\begin{algorithm}[SSSP$(V,E,start)$] \label{alg:sssp}
\alginput{Vertex set $V$, edge set $E$, starting vertex $start$}
\algoutput{Distance and parent set for every vertex $cur$}
\begin{algorithmic}[1]
	\State $toupdate(0) \set (start,0,start)$
	\For{every $i \in 0\dots|V|-1$}
		\For{every tuple $t \in toupdate(i)$} \label{alg:sssp:update}
			\For{every edge $e \mid e.src = t.id$}
		 		\State $local \set e.val + t.val$
		 		\If{$local < toupdate(i+1,e.dest).val$} \label{alg:sssp:single}
		 			\State $toupdate(i+1,dest) \set (local,e.src)$
		 		\EndIf
			\EndFor
		\EndFor
		\For{every tuple $t \in toupdate(i+1)$}
		 	\If{$t.val < cur(t.id).val$}
		 		\State $cur(t.id) \set (t.val,t.parent)$
		 	\EndIf
		\EndFor
	\EndFor
\end{algorithmic}
\end{algorithm}

\begin{description}
\item edge: (src,dest,val). The edges of the graph.
\item cur: id -> (val,parent). The intermediate SSSP results.
\item toupdate: iter -> (id -> (val,parent)). The set of updates.
\end{description}

Changes from the standard Bellman-Ford algorithm:

\begin{description}
\item Line~\ref{alg:sssp:update}: We only check the vertices that have been updated in the last iteration.
\item Line~\ref{alg:sssp:single}: At each iteration, we update a given vertex only one time. This means the toupdate set cannot contain multiple records for the same vertex which requires the comparison with the existing value.
\end{description}

This is not a 1-to-1 pseudocode for the implementation since we don't compare the `toupdate` table records one by one but calculate the overall minimum. In addition, the comparison with `cur` values take place earlier to reduce the number of tuples in the `toupdate` table.

