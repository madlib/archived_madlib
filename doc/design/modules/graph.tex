% When using TeXShop on the Mac, let it know the root document. The following
% must be one of the first 20 lines.
% !TEX root = ../design.tex

% Licensed to the Apache Software Foundation (ASF) under one
% or more contributor license agreements.  See the NOTICE file
% distributed with this work for additional information
% regarding copyright ownership.  The ASF licenses this file
% to you under the Apache License, Version 2.0 (the
% "License"); you may not use this file except in compliance
% with the License.  You may obtain a copy of the License at

%   http://www.apache.org/licenses/LICENSE-2.0

% Unless required by applicable law or agreed to in writing,
% software distributed under the License is distributed on an
% "AS IS" BASIS, WITHOUT WARRANTIES OR CONDITIONS OF ANY
% KIND, either express or implied.  See the License for the
% specific language governing permissions and limitations
% under the License.

\chapter[Graph]{Graph}

\begin{moduleinfo}
\item[Author] \href{mailto:okislal@pivotal.io}{Orhan Kislal}
\item[History]
	\begin{modulehistory}
		\item[v0.1] Initial version, SSSP only.
		\item[v0.2] Graph Framework, SSSP implementation details.
	\end{modulehistory}
\end{moduleinfo}


% Abstract. What is the problem we want to solve?

This module implements various graph algorithms that are used in a number of
applications such as social networks, telecommunications and road networks.

\section{Graph Framework} \label{sec:graph:fw}

MADlib graph representation depends on two structures, a \emph{vertex} table
and an \emph{edge} table. The vertex table has to have a column of vertex ids.
The edge table has to have 2 columns: source vertex id, destination vertex id.
For most algorithms an edge weight column is required as well. The
representation assumes a directed graph, an edge from $x$ to $y$ does
\emph{not} guarantee the existence of an edge from $y$ to $x$. Both of the
tables may have additional columns as required. Multi-edges (multiple edges
from a vertex to the same destination) and loops (edge from a vertex to
itself) are allowed. This representation does not impose any ordering of
vertices or edges. An example graph is given in Figure~\ref{sssp:example} and
its representative tables are given in Table~\ref{sssp:rep}.

\begin{figure}[h]
	\centering
	\includegraphics[width=0.9\textwidth]{figures/graph_example.pdf}
\caption{A sample graph}
\label{sssp:example}
\end{figure}

\begin{table}
  \begin{tabular}{| c | }
    \hline
    vid \\ \hline
    0 \\ \hline
    1 \\ \hline
    2 \\ \hline
    3 \\ \hline
    4 \\ \hline
    5 \\ \hline
    6 \\ \hline
    7 \\
    \hline
  \end{tabular}
  \quad
  \begin{tabular}{| c | c | c |}
    \hline
    src & dest & weight \\ \hline
    0 & 1 & 1 \\ \hline
    0 & 2 & 1 \\ \hline
    0 & 4 & 10 \\ \hline
    1 & 2 & 2 \\ \hline
    1 & 3 & 10 \\ \hline
    1 & 5 & 1 \\ \hline
    2 & 3 & 1 \\ \hline
    2 & 5 & 1 \\ \hline
    2 & 6 & 3 \\ \hline
    3 & 0 & 1 \\ \hline
    5 & 6 & 1 \\ \hline
    6 & 7 & 1 \\
    \hline
  \end{tabular}
  \caption{Graph representation of vertices (left) and edges(right) in the
  database}
  \label{sssp:rep}
\end{table}


\section{Single Source Shortest Path} \label{sec:graph:sssp}

Given a graph and a source vertex, single source shortest path (SSSP)
algorithm finds a path for every vertex such that the sum of the weights of
its constituent edges is minimized.

Shortest path is defined as follows. Let $e_{i,j}$ be the edge from vertex $i$
to vertex $j$ and $w_{i,j}$ be its weight. Given a graph G, the shortest path
from $s$ to $d$ is $P = (v_1, v_2 \dots, v_n)$ (where $v_1=s$ and $v_n=d$)
that over all possible $n$ minimizes the sum $ \sum _{i=1}^{n-1}f(e_{i,i+1})$.

% \subsection{Bellman Ford Algorithm}

Bellman-Ford Algorithm \cite{bellman1958routing,ford1956network} is based on
the following idea: We start with a naive approximation for the cost of
reaching every vertex. At each iteration, these values are refined based on
the edge list and the existing approximations. If there are no refinements at
any given step, the algorithm returns the calculated results. If the algorithm
does not converge in $|V|-1$ iterations, this indicates the existence of a
negative cycle in the graph.


\begin{algorithm}[SSSP$(V,E,start)$] \label{alg:sssp}
\alginput{Vertex set $v$, edge set $E$, starting vertex $start$}
\algoutput{Distance and parent set for every vertex $cur$}
\begin{algorithmic}[1]
	\State $toupdate(0) \set (start,0,start)$
	\For{every $i \in 0\dots|V|-1$}
		\For{every tuple $t \in toupdate(i)$} \label{alg:sssp:update}
			\For{every edge $e \mid e.src = t.id$}
		 		\State $local \set e.val + t.val$
		 		\If{$local < toupdate(i+1,e.dest).val$} \label{alg:sssp:single}
		 			\State $toupdate(i+1,dest) \set (local,e.src)$
		 		\EndIf
			\EndFor
		\EndFor
		\For{every tuple $t \in toupdate(i+1)$}
		 	\If{$t.val < cur(t.id).val$}
		 		\State $cur(t.id) \set (t.val,t.parent)$
		 	\EndIf
		\EndFor
	\EndFor
\end{algorithmic}
\end{algorithm}

\begin{description}
\item edge: (src,dest,val). The edges of the graph.
\item cur: id -> (val,parent). The intermediate SSSP results.
\item toupdate: iter -> (id -> (val,parent)). The set of updates.
\end{description}

Changes from the standard Bellman-Ford algorithm:

\begin{description}
\item Line~\ref{alg:sssp:update}: We only check the vertices that have been
updated in the last iteration.
\item Line~\ref{alg:sssp:single}: At each iteration, we update a given vertex
only one time. This means the toupdate set cannot contain multiple records
for the same vertex which requires the comparison with the existing value.
\end{description}

This is not a 1-to-1 pseudocode for the implementation since we don't compare
the `toupdate` table records one by one but calculate the overall minimum. In
addition, the comparison with `cur` values take place earlier to reduce the
number of tuples in the `toupdate` table.

\subsection{Implementation Details}

In this section, we discuss the MADlib implementation of the SSSP algorithm
in depth.

\begin{algorithm}[SSSP$(V,E,start)$] \label{alg:sssp:high}
\begin{algorithmic}[1]
	\Repeat
		\State Find Updates
		\State Apply updates to the output table
	\Until {There are no updates}
\end{algorithmic}
\end{algorithm}

The implementation consists of two SQL blocks that are called sequentially
inside a loop. We will follow the example graph at Figure~\ref{sssp:example}
with the starting point as $v_0$. The very first update on the output table is
the source vertex. Its weight is $0$ and its parent is itself ($v_0$). After
this initialization step, the loop starts with Find Updates (the individual
updates will be represented with <dest,value,parent> format). Looking at the
example, it is clear that the updates should be <1,1,0>, <2,1,0> and <4,10,0>.
We will assume this iteration is already completed and look how the next
iteration of the algorithm works to explain the implementation details.

\begin{algorithm}[Find Updates$(E,old\_update,out\_table)$]
\label{alg:sssp:findu}
\begin{lstlisting}
INSERT INTO new_update
	SELECT DISTINCT ON (y.id) y.id AS id,
		y.val AS val,
		y.parent AS parent
	FROM out_table INNER JOIN (
			SELECT edge_table.dest AS id, x.val AS val, old_update.id AS parent
			FROM old_update
				INNER JOIN edge_table
				ON (edge_table.src = old_update.id)
				INNER JOIN (
					SELECT edge_table.dest AS id,
						min(old_update.val + edge_table.weight) AS val
					FROM old_update INNER JOIN
						edge_table AS edge_table ON
						(edge_table.src=old_update.id)
					GROUP BY edge_table.dest
				) x
				ON (edge_table.dest = x.id)
			WHERE ABS(old_update.val + edge_table.weight - x.val) < EPSILON
		) AS y ON (y.id = out_table.vertex_id)
	WHERE y.val<out_table.weight
\end{lstlisting}
\end{algorithm}

The Find Updates query is constructed in 4 levels of subqueries: \emph{find
values, find parents, eliminate duplicates and ensure improvement}.

\begin{itemize}

\item We begin our analysis at the innermost subquery, emph{find values}
(lines 11-16). This subquery takes a set of vertices (in the table
$old_update$) and finds the reachable vertices. In case a vertex is reachable
by multiple vertices, only the path that has the minimum cost is considered
(hence the name find values). There are two important points to note:
	\begin{itemize}
	\item The input vertices need the value of their path as well.
		\begin{itemize}
		\item In our example, both $v_1$ and $v_2$ can reach $v_3$. We would
		have to use $v_2$ -> $v_3$ edge since that gives the lowest possible
		path value.
		\end{itemize}
	\item The subquery is aggregating the rows using the $min$ operator for
	each destination vertex and  unable to return the source vertex at the
	same time to use as the parent value.
		\begin{itemize}
		\item We know the value of $v_3$ should be $2$ but we cannot know
		its parent ($v_2$) at the same time.
		\end{itemize}
	\end{itemize}

\item The \emph{find parents} subquery is designed to solve the
aforementioned limitation. We combine the result of \emph{find values} with
$edge$ and $old\_update$ tables (lines 7-10) and get the rows that has the
same minimum value.
	\begin{itemize}
	\item Note that, we would have to tackle the problem of tie-breaking.
		\begin{itemize}
		\item Vertex $v_5$ has two paths leading into: <5,2,1> and <5,2,2>.
		The inner subquery will return <5,2> and it will match both of these
		edges.
		\end{itemize}
	\item It is redundant to keep both of them in the update list as that
	would require updating the same vertex multiple times in a given
	iteration.
	\end{itemize}

\item At this level, we employ the \emph{eliminate duplicates} subquery. By
using the $DISTINCT$ clause at line 2, we allow the underlying system to
accept only a single one of them.

\item Finally, we introduce the \emph{ensure improvement} subquery to make
sure these updates are actually leading us to shortest paths. Line 21 ensures
that the values stored in the $out\_table$ does not increase and the solution
does not regress throughout the iterations.
\end{itemize}

Applying updates is straightforward as the values and the associated parent
values are replaced using the $new\_update$ table. After this operation is
completed the $new\_update$ table becomes $old\_update$ for the next iteration
of the algorithm.

Please note that, for ideal performance, \emph{vertex} and \emph{edge} tables
should be distributed on \emph{vertex id} and \emph{source id} respectively.

