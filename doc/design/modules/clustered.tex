
!TEX root = ../design.tex

\chapter[Clustered Standard Errors]{Clustered Standard Errors}
\begin{moduleinfo}
\item[History]
  \begin{modulehistory}
    \item[v0.1] Initial version
    \end{modulehistory}
\end{moduleinfo}

% Abstract. The problem 

Adjusting standard errors for clustering can be important. For
example, replicating a dataset 100 times should not increase the
precision of parameter estimates. However, performing this procedure
with the IID assumption will actually do this. Another example is in
economics of education research, it is reasonable to expect that the
error terms for children in the same class are not
independent. Clustering standard errors can correct for this.  

\section{Overview of Clustered Standard Errors}

Assume that the data can be separated into $m$ clusters. Usually this
can be down by grouping the data table according to one or multiple
columns.

The estimator has a similar form to the usual sandwich estimator
\begin{equation}
  S(\vec{c}) = B(\vec{c}) M(\vec{c}) B(\vec{c})
\end{equation}

The bread part is the same as sandwich estimator
\begin{eqnarray}
  B(\vec{c}) & = & \left(-\sum_{i=1}^{n} H(y_i, \vec{x}_i,
    \vec{c})\right)^{-1}\\
  & = & \left(-\sum_{i=1}^{n}\frac{\partial^2 l(y_i, \vec{x}_i,
      \vec{c})}{\partial c_\alpha \partial c_\beta}\right)^{-1}
\end{eqnarray}
where $H$ is the hessian matrix, which is the second derivative of the
target function 
\begin{equation}
  L(\vec{c}) = \sum_{i=1}^n l(y_i, \vec{x}_i, \vec{c})\ .
\end{equation}

The meat part is different
\begin{equation}
  M(\vec{c}) = \bf{A}^T\bf{A}
\end{equation}
where the $m$-th row of $\bf{A}$ is 
\begin{equation}
  A_m = \sum_{i\in G_m}\frac{\partial
      l(y_i,\vec{x}_i,\vec{c})}{\partial \vec{c}}
\end{equation}
where $G_m$ is the set of rows that belong to the same cluster.

\section{Implementation}

We can compute the quantities of $B$ and $A$ for each cluster during one scan through
the data table in an aggregate function. Then sum over all clusters to
the full $B$ and $A$ in the outside of the aggregate function. At last, the matrix mulplitications
are
done in a separate function on master.  
