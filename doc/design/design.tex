\documentclass[letterpaper,11pt]{scrreprt}
\usepackage[american]{babel}
\usepackage[latin1]{inputenc}
\usepackage[T1]{fontenc}
\usepackage[top=1in,bottom=1in,left=1in,right=1in]{geometry}

\usepackage{amsmath}
\usepackage{amssymb}

\usepackage[hyperref]{ntheorem}
\usepackage[
	bookmarks,
	colorlinks=false,
	linkcolor=blue,
	citecolor=blue,
	pagebackref=false,
	pdftitle={MADlib Design Document},
	pdfauthor={Florian Schoppmann},
	pdfsubject={},
	pdfkeywords={}
]{hyperref}
\usepackage{csquotes}                  % Strongly recommended for biblatex
\usepackage[
	backend=bibtex,
	maxnames=2,
	maxbibnames=20,
	firstinits=true
]{biblatex}
\usepackage{scrpage2}                  % Headers and footers
\usepackage{color}                     % Colors, possibly only for \todo
\usepackage{enumitem}                  % enumerate environment
\usepackage{ctable}
\usepackage{tabularx}
\usepackage{xspace}                    % Correct spaces after \newcommand definitions
\usepackage[noend]{algpseudocode}      % algorithm environment
\usepackage{listings}                  % Code snippets
\usepackage{bbding}


% BEGIN Doc Layout
	\allowdisplaybreaks[3]

	\pagestyle{scrheadings}
	\automark[chapter]{section}

	\setkomafont{disposition}{\normalcolor\bfseries}
	\setkomafont{descriptionlabel}{\bfseries}
	\setkomafont{captionlabel}{\usekomafont{disposition}}

	\setlength{\arrayrulewidth}{.5pt}
	\numberwithin{equation}{section}
	\renewcommand{\theenumi}{\roman{enumi}}
	\renewcommand{\labelenumi}{\theenumi)}

	\newcommand{\otoprule}{\midrule[\heavyrulewidth]}

	\setcounter{secnumdepth}{3}

	\makeatletter
	% Algorithms are expected to have an optional argument of form
	% FunctionName$(ArgumentList)$, e.g., DiscreteSample$(A, w)$
	\def\internal@funcName#1$(#2)${#1}
	\newcommand\funcName[1]{\internal@funcName #1}
	\newtheoremstyle{algorithm}
		{\item[\rlap{\vbox{\hbox{\hskip\labelsep \theorem@headerfont
			##1\ ##2\theorem@separator}\hbox{\strut}}}]}%
		{\item[\rlap{\vbox{\hbox{\hskip\labelsep {\theorem@headerfont
			##1}\ \normalfont\texttt{##3}{\theorem@headerfont\theorem@separator}}\hbox{\strut}}}]%
			\def\@currentlabel{\texttt{\funcName{##3}}}}
	\makeatother

	\makeatletter
	% Also display JSTOR in small caps
	% http://sourceforge.net/tracker/index.php?func=detail&aid=3152938&group_id=244752&atid=1126006
	\DeclareFieldFormat{eprint:arxiv}{%
	  \textsc{arXiv}\addcolon
	  \ifhyperref
	    {\href{http://arxiv.org/\abx@arxivpath/#1}{%
	       \nolinkurl{#1}%
	       \iffieldundef{eprintclass}
		 {}
		 {\addspace\texttt{\mkbibbrackets{\thefield{eprintclass}}}}}}
	    {\nolinkurl{#1}
	     \iffieldundef{eprintclass}
	       {}
	       {\addspace\texttt{\mkbibbrackets{\thefield{eprintclass}}}}}}
	\DeclareFieldFormat{eprint:jstor}{%
	  \mkbibacro{JSTOR}\addcolon\space
	  \ifhyperref
	    {\href{http://www.jstor.org/stable/#1}{\nolinkurl{#1}}}
	    {\nolinkurl{#1}}}
	% Some conferences do not have DOIs for their papers, but they do get
	% IDs in the ACM Digital Library. E.g., SODA papers.
	\DeclareFieldFormat{eprint:acm}{%
	  \mkbibacro{ACM}\addcolon\space
	  \ifhyperref
	    {\href{http://dl.acm.org/citation.cfm?id=#1}{\nolinkurl{#1}}}
	    {\nolinkurl{#1}}}
	\makeatother
	
	\newlist{moduleinfo}{description}{2}
	\setlist[moduleinfo]{style=multiline,labelindent=\leftmargini,leftmargin=3cm,rightmargin=\leftmargini,font=\bfseries}
	\newlist{modulehistory}{description}{2}
	\setlist[modulehistory]{style=multiline,leftmargin=1.1cm}
% END Doc Layout

% BEGIN General Definitions
	\newcommand{\todo}[1]{\textbf{\color{red}#1}}

	\newcommand{\specialcell}[3][t]{%
		\begin{tabular}[#1]{@{}#2@{}}#3\end{tabular}}

	% BEGIN Mathematical Definition
		% Space (only) in displaymath (e.g., between mathematical expression and punctuation mark)
		\newcommand{\SiM}{\mathchoice{\,}{}{}{}}
	% END Mathematical Operators

	% BEGIN URLs
		\newcommand{\mailto}[1]{\href{mailto:#1}{\nolinkurl{#1}}}
		\newcommand{\doi}[1]{DOI: \href{http://dx.doi.org/#1}{\nolinkurl{#1}}}
	% END URLs

	\makeatletter
	% BEGIN Mathematical Definitions
		% BEGIN Set Symbols
			\newcommand{\setsymbol}[1]{\mathbb{#1}}
			\newcommand{\N}{\@ifstar{\setsymbol{N}_0}{\setsymbol{N}}}
			\newcommand{\R}{\setsymbol{R}}
		    \newcommand{\Nupto}{\@ifstar{\Nupto@star}{\Nupto@nostar}}
		    \newcommand{\Nupto@star}[1]{[#1]_0}
		    \newcommand{\Nupto@nostar}[1]{[#1]}
		% END Set Symbols
		\renewcommand{\vec}[1]{\ensuremath{\boldsymbol{#1}}}
	% END Mathematical Definitions
	\makeatother

	\renewcommand{\vec}[1]{\ensuremath{\boldsymbol{#1}}}
	\newcommand{\enumref}[1]{(\ref{#1})}

	\makeatletter
	\newcommand{\symlabel}[2]{\def\@currentlabel{\texttt{#1}}\texttt{#1}\label{#2}}
	\makeatother
	
	\newcommand{\Warning}[1]{\marginpar[\HandRight]{\HandLeft}\textbf{#1}}

	% BEGIN Algorithms
	\theoremstyle{algorithm}
	\theorembodyfont{\upshape}
	\newtheorem{algorithm}{Algorithm}[section]

	\newlength{\alglabelwidth}
	\newcommand{\alginput}[1]{%
		\par\noindent%
		\settowidth{\alglabelwidth}{\emph{Output:}}%
		\makebox[\alglabelwidth][l]{\emph{Input:}} \begin{tabular}[t]{l} #1 \end{tabular}}
	\newcommand{\algoutput}[1]{%
		\par\noindent%
		\settowidth{\alglabelwidth}{\emph{Output:}}%
		\makebox[\alglabelwidth][l]{\emph{Output:}} \begin{tabular}[t]{l} #1 \end{tabular}}
	\newcommand{\algprecond}[1]{%
		\par\noindent\textit{Initialization/Precondition: #1}}

	\newcommand{\set}{\leftarrow}
	\DeclareMathOperator{\random}{random}
	\newcommand{\dist}{\ensuremath{\mathit{dist}}}
	\newcommand{\List}{\mathrm{List}}
	\newcommand{\Sample}{\mathit{Sample}}
	\algblockdefx[With]{With}{EndWith}%
		[1]{\textbf{with} #1 \textbf{do}}%
		[0]{End}
	\algnotext[With]{EndWith}
	% END Algorithms

	% BEGIN lstlisting environments
	\lstset{
		basicstyle=\ttfamily\scriptsize,       % the size of the fonts that are used for the code
		numbers=left,                   % where to put the line-numbers
		numberstyle=\ttfamily,      % the size of the fonts that are used for the line-numbers
		%aboveskip=0pt,
		%belowskip=0pt,
		stepnumber=1,                   % the step between two line-numbers. If it is 1 each line will be numbered
		%numbersep=10pt,                  % how far the line-numbers are from the code
		breakindent=0pt,
		firstnumber=1,
		%backgroundcolor=\color{white},  % choose the background color. You must add \usepackage{color}
		showspaces=false,               % show spaces adding particular underscores
		showstringspaces=false,         % underline spaces within strings
		showtabs=false,                 % show tabs within strings adding particular underscores
		frame=leftline,
		tabsize=2,  		% sets default tabsize to 2 spaces
		captionpos=b,   		% sets the caption-position to bottom
		breaklines=false,    	% sets automatic line breaking
		breakatwhitespace=true,    % sets if automatic breaks should only happen at whitespace
		%escapeinside={\%}{)}          % if you want to add a comment within your code
		columns=fixed,
		basewidth=0.52em,
		% are you fucking kidding me lstlistings?  who puts the line numbers outside the margin?
		xleftmargin=6mm,
		xrightmargin=-6mm,
		numberblanklines=false,
		language=C++,
		morekeywords={table,scratch,channel,interface,periodic,bloom,state,bootstrap,morph,monotone,lset,lbool,lmax,lmap}
	}
	% END lstlisting environments
	
	\lstnewenvironment{sql}[1][]{\lstset{language=SQL,gobble=4,emphstyle=\textit,#1}}{}
% END General Definitions

\bibliography{../literature.bib}

% BEGIN Preamble
\title{%
	MADlib Design Document%
}

\begin{document}

\maketitle

\tableofcontents

% When using TeXShop on the Mac, let it know the root document. The following must be one of the first 20 lines.
% !TEX root = ../design.tex

\chapter{Sampling}

\begin{moduleinfo}
\item[Author] \href{mailto:Florian.Schoppmann@emc.com}{Florian Schoppmann}
\item[History]
	\begin{modulehistory}
		\item[v0.5] Initial revision
	\end{modulehistory}
\end{moduleinfo}

\section{Sampling without Replacement} \label{sec:SampingWOReplacement}

Given a list of known size $n$ through that we can iterate with arbitrary increments, sampling $m$ elements without replacement can be implemented in time $O(m)$, i.e., proportional to the sample size and independent of the list size \cite{V84a}. Even if the list size $n$ is unknown in advance, sampling can still be implemented in time $O(m(1 + \log \frac nm))$ \cite{V85a}.

While being able to iterate through a list with arbitrary increments might seem like a very modest requirement, it is still not always an option in real-world databases (e.g., in PostgreSQL). It is therefore important to also consider more constrained algorithms.

\subsection{Probabilistic Sampling}

Probabilistic sampling selects each element in the list with probability $p$. Hence, the sample size is a random variable with Binomial distribution $B(n, p)$ and expectation $np$. The standard deviation is $\sqrt{np(1 - p)}$, i.e., approximately $\sqrt{np}$ if $p$ is small. In many applications a fixed sample size $m$ is needed, however. In this case, we could choose $p$ slightly larger than $m/n$, so that with high probability at least $m$ items are selected. Items in excess of $m$ are then discarded.

\subsubsection{Formal Description}

In the following, we discuss how to choose $p$ so that with high probability at least $m$ elements are sampled, but also not ``much'' more than $m$ (in fact, only $O(\sqrt m)$ more in expectation).

In mathematical terms: What is a lower bound on the probability $p$ so that for a random variable $X \sim B(n,p)$ we have that $\Pr[X < m] \leq \epsilon$? We use the Chernoff bound for a fairly good estimate. It says
%
\begin{align*}
	\Pr[X < (1 - \delta) \cdot \mu] \leq \exp\left( \frac{-\delta^2}{2} \cdot \mu \right)
	\SiM,
\end{align*}
where $\mu = np$ is the expectation of $X$, and $\delta \geq 0$. We set $m = (1 - \delta) \cdot \mu$, or equivalently $\delta = \frac{\mu - m}{\mu}$.
%
This yields
\begin{align}
	\Pr[X < m] \leq \exp\left( \frac{-(\mu - m)^2}{2 \mu} \right)
	\SiM. \label{eq:SamplingWOReplacent:GP:2}
\end{align}
%
We want the right-hand side of \eqref{eq:SamplingWOReplacent:GP:2} to be bounded by $\epsilon$ from above. Rearranging this gives
\begin{align*}
	\mu \geq m - \ln(\epsilon) + \sqrt{\ln^2(\epsilon) - 2 m \ln(\epsilon)}
	\SiM.
\end{align*}
Since $p = \mu / n$, this immediately translates into a lower bound for $p$. For instance, suppose we require $\epsilon = 10^{-6}$, i.e., we want the probability of our sample being too small to be less than one in a million. $\ln(10^{-6}) \approx -13.8$, so we could choose
\begin{align*}
	p \geq \frac{m + 14 + \sqrt{196 + 28m}}{n}
	\SiM.
\end{align*}

Note that the bound on $\mu$ does not depend on $n$. So in expectation, only $O(m + \sqrt m)$ items are selected. At the same time, at least $m$ items are selected with very high probability.

\subsubsection{Implementation in SQL}

In real-world DBMSs, probabilistic sampling has the advantage that it is trivially data-parallel. Discarding excessive items can be done using the well-known \texttt{ORDER BY random() LIMIT} idiom. Tests show that PostgreSQL is very efficient in doing the sorting (today's CPUs can easily sort 1 million numbers in less than a couple hundred milliseconds). In fact, the sorting cost is almost not measurable if the sample size is only at the scale of several million or less. Since \texttt{ORDER BY random() LIMIT} is an often-used idiom, there is also hope that advanced optimizers might give it special treatment. Put together, in order to sample $m$ random rows uniformly at random, we write:
\begin{sql}
	SELECT * FROM list WHERE random() < p ORDER BY random() LIMIT m
\end{sql}
If necessary, checks can be added that indeed $m$ rows have been selected.


\subsection{Generating a Random Variate According to a Discrete Probability Distribution}

In practice, probability distributions are often induced by weights (that are not necessarily normalized to add up to 1). The following algorithm is a special case of the ``unequal probability sampling plan'' proposed by \textcite{C82a}. Its idea is very similar to reservoir sampling \cite{MB83a}.

\subsubsection{Formal Description}

\begin{algorithm}[WeightedSample$(A, w)$] \label{alg:WeightedSample}
\alginput{Finite set $A$, Mapping $w$ of each element $a \in A$ to its weight $w[a] \geq 0$}
\algoutput{Random element $\Sample \in A$ sampled according to distribution induced by $w$}
\begin{algorithmic}[1]
	\State $W \set 0$
	\For{$a \in A$}
		\State $W \set W + w[a]$ \label{alg:WeightedSample:UpdateWeight}
		\With{probability $\frac{w[a]}{W}$} \label{alg:WeightedSample:Prob}
			\State $\Sample \set a$ \label{alg:WeightedSample:SetSample}
		\EndWith
	\EndFor
\end{algorithmic}
\end{algorithm}

\begin{description}
	\item[Runtime] $O(n)$, single-pass streaming algorithm
	\item[Space] $O(1)$, constant memory
	\item[Correctness]
		Let $a_1, \dots, a_n$ be the order in which the algorithm processes the elements. Denote by $\Sample_t$ the value of $\Sample$ at the end of iteration $t \in \Nupto n$. We prove by induction over $t$ that it holds for all $i \in \Nupto t$ that $\Pr[\Sample_t = a_i] = \frac{w[a_i]}{W_t}$ where $W_t := \sum_{j=1}^t w[a_j]$.

		The base case $t = 1$ holds immediately by lines~\ref{alg:WeightedSample:Prob}--\ref{alg:WeightedSample:SetSample}. To see the induction step $t - 1 \to t$, note that $\Pr[\Sample_t = a_t] = \frac{w[a_t]}{W_t}$ (again by lines~\ref{alg:WeightedSample:Prob}--\ref{alg:WeightedSample:SetSample}) and that for all $i \in \Nupto{t-1}$
		\begin{align*}
			\Pr[\Sample_t = a_i]
			=	\Pr[\Sample_t \neq a_t] \cdot \Pr[\Sample_{t-1} = a_i]
			\stackrel{\text{IH}}{=}
				\left(1 - \frac{w[a_t]}{W_t}\right) \cdot \frac{w[a_i]}{W_{t-1}}
			=	\frac{w[a_i]}{W_t}
			\SiM.
		\end{align*}
	\item[Scalability]
		The algorithm can easily be transformed into a divide-and-conquer algorithm, as shown in the following.
\end{description}

\begin{algorithm}[RecursiveWeightedSample$(A_1, A_2, w)$] \label{alg:RecursiveWeightedSample}
\alginput{Disjoint finite sets $A_1, A_2$, Mapping $w$ of each element $a \in A_1 \cup A_2$ to its weight $w[a] \geq 0$}
\algoutput{Random element $\Sample \in A_1 \cup A_2$ sampled according to distribution induced by $w$}
\begin{algorithmic}[1]
	\State $\tilde A \set \emptyset$
	\For{$i = 1,2$}
		\State $\Sample_i \set \texttt{WeightedSample}(A_i, w)$
		\State $\tilde A \set \tilde A \cup \{ \Sample_i \}$
		\State $\tilde w[\Sample_i] \set \sum_{a \in A_i} w[a]$
	\EndFor
	\State $\Sample \set \texttt{WeightedSample}(\tilde A, \tilde w)$
\end{algorithmic}
\end{algorithm}

\paragraph{Correctness}

Define $W_i := \sum_{a \in A_i} w[a]$. Let $a \in A_i$ be arbitrary. Then $\Pr[\Sample = a] = \Pr[\Sample_i = a] \cdot \Pr[\Sample \in A_i] = \frac{w[a]}{W_i} \cdot \frac{W_i}{W} = \frac{w[a]}{W}.$

\paragraph{Numerical Considerations}

\begin{itemize}
	\item When Algorithm~\ref{alg:WeightedSample} is used for large sets $A$, line~\ref{alg:WeightedSample:UpdateWeight} will eventually add two numbers that are very different in size. Compensated summation should be used \cite{ORO05a}.
\end{itemize}

\subsubsection{Implementation as User-Defined Aggregate}

Algorithm~\ref{alg:WeightedSample} is implemented as the user-defined aggregate \symlabel{weighted\_sample}{sym:weighted_sample}. Data-parallelism is implemented as in Algorithm~\ref{alg:RecursiveWeightedSample}.

\paragraph{Input} The aggregate expects the following arguments:

\begin{center}
	\begin{tabularx}{\linewidth}{lXl}
		\toprule%
		\textbf{Column} & \textbf{Description} & \textbf{Type}
		\\\otoprule
		\texttt{value} &
		Row identifier, each row corresponds to an $a \in A$. There is no need to enforce uniqueness. If a value occurs multiple times, the probability of sampling this value is proportional to the sum of its weights. &
		\textit{generic}
		\\\midrule
		\texttt{weight} &
		weight for row, corresponds to $w[a]$ &
		floating-point
		\\\bottomrule
	\end{tabularx}
\end{center}
%
While it would be desirable to define a user-defined aggregate with a first argument of generic type, this would require a generic transition type (see below). Unfortunately, this is currently not supported in PostgreSQL and Greenplum. However, the internal C++ accumulator type is a generic template class, i.e., only the SQL interface contains redundancies.

\paragraph{Output}

Value of column \texttt{id} in row that was selected.

\paragraph{Components}

\begin{itemize}
	\item Transition State:
		\begin{center}
			\begin{tabularx}{\linewidth}{lXl}
				\toprule%
				\textbf{Field Name} & \textbf{Description} & \textbf{Type}
				\\\otoprule
				\texttt{weight\_sum} &
				corresponds to $W$ in Algorithm~\ref{alg:WeightedSample} &
				floating-point
				\\\midrule
				\texttt{sample} &
				corresponds to $\Sample$ in Algorithm~\ref{alg:WeightedSample}, takes value of column \texttt{value} &
				\textit{generic}
				\\\bottomrule
			\end{tabularx}
		\end{center}
	\item Transition Function (\texttt{state, id, weight}): Lines~\ref{alg:WeightedSample:UpdateWeight}--\ref{alg:WeightedSample:SetSample} of Algorithm~\ref{alg:WeightedSample}
	\item Merge Function (\texttt{state1, state2}): It is enough to call the transition function with arguments (\texttt{state1, state2.sample\_id, state2.weight\_sum})
\end{itemize}

\paragraph{Tool Set}

While the user-defined aggregate is simple enough to be implemented in plain SQL, we choose a C++ implementation for performance. \todo{In the future, we want to use compensated summation. (Not documented yet.)}

% When using TeXShop on the Mac, let it know the root document. The following must be one of the first 20 lines.
% !TEX root = ../design.tex

\chapter{Matrix Operations}

% Abstract. What is the problem we want to solve?
While dense and sparse matrices are native objects for MADlib, they are not part of the SQL standard. It is therefore essential to provide bridges between SQL types and MADlib types, as well provide a ground set of primitive functions that can be used in SQL.

\section{Constructing Matrices}

\subsection{Construct a matrix from columns stored as tuples} \label{sec:matrix:matrixAgg}

Let $X = ( x_1, \dots, x_n ) \subset \R^m$. \symlabel{matrix\_agg}{sym:matrixAgg}$(X)$ returns the matrix $(x_1 \dots x_n) \in \R^{m \times n}$.

\subsubsection{Implementation as User-Defined Aggregate}

\begin{center}
	\begin{tabular}{rlll}
		\toprule%
		& \textbf{Name} & \textbf{Description} & \textbf{Type}
		\\\otoprule
		In &
		\texttt{x} &
		Vector $x_i \in \R^m$ &
		floating-point vector
		\\\midrule
		Out & &
		Matrix $M = (x_1 \dots x_n) \in \R^{m \times n}$ &
		floating-point matrix
		\\\bottomrule
	\end{tabular}
\end{center}


\section{Norms and Distances}

\subsection{Column in a matrix that is closest to a given vector} \label{sec:matrix:closestColumn}

Let $M \in \R^{m \times n}$, $x \in \R^m$, and $\dist$ be a metric. \symlabel{closest\_column}{sym:closestColumn}$(M, x, \dist)$ returns a tuple $(i,d)$ so that $d = \dist(x, M_i) = \min_{j \in \Nupto n} \dist(x, M_j)$ where $M_j$ denotes the $j$-th column of $M$.

\subsubsection{Implementation as User-Defined Function}

\begin{center}
	\begin{tabular}{rlll}
		\toprule%
		& \textbf{Name} & \textbf{Description} & \textbf{Type}
		\\\otoprule
		In &
		\texttt{M} &
		Matrix $M \in \R^{m \times n}$ &
		floating-point matrix
		\\\midrule
		In &
		\texttt{x} &
		Vector $x \in \R^m$ &
		floating-point vector
		\\\midrule
		In &
		\texttt{dist} &
		Metric to use &
		function
		\\\midrule
		Out &
		\texttt{column\_id} &
		index $i$ of the column of $M$ that is closest to $x$ &
		integer
		\\\midrule
		Out &
		\texttt{distance} &
		$\dist(x, M_i)$ &
		floating-point
		\\\bottomrule
	\end{tabular}
\end{center}

% When using TeXShop on the Mac, let it know the root document. The following must be one of the first 20 lines.
% !TEX root = ../design.tex

\chapter[Clustering (k-means et al.)]{Clustering ($k$-Means et al.)}

\begin{moduleinfo}
\item[Author] \href{mailto:Florian.Schoppmann@emc.com}{Florian Schoppmann} (version 0.5 only)
\item[History]
	\begin{modulehistory}
		\item[v0.5] Initial revision of design document, complete rewrite of module, arbitrary user-specifyable distance and recentering functions
		\item[v0.3] Multiple seedings methods (kmeans++, random, user-specified list of centroids), multiple distance functions (corresponding recentering function hard-coded), simplified silhouette coefficient as goodness-of-fit measure
		\item[v0.1] Initial version, always use kmeans++ for seeding
	\end{modulehistory}
\end{moduleinfo}


% Abstract. What is the problem we want to solve?
Clustering refers to the problem of partitioning a set of objects into homogeneous subsets, i.e., such that objects in the same group have \emph{similar} properties. Arguably the best-known clustering problem is \emph{$k$-means}. Here, one is given $n$ points $x_1, \dots, x_n \in \R^d$, and the goal is to position $k$ centroids $c_1, \dots, c_k \in \R^d$ so that the sum of squared Euclidean distances between each point and its closest centroid is minimized. A cluster is identified by its centroid and consists of all points for which this centroid is closest. Formally, we wish to minimize the following objective function:
\begin{gather*}
    (c_1, \dots, c_k) \mapsto \sum_{i=1}^n \min_{j=1}^k \dist(x_i, c_j) \SiM,
\end{gather*}
where $\dist(x,y) := \| x - y \|_2^2$. A straightforward generalization of the above minimization problem is to choose a different metric $\dist$. Strictly speaking, this is no longer $k$-means; for instance, choosing $\dist(x,y) = \| x - y\|_1$ yields the $k$-median problem instead.

Despite certain reservations, we follow the practice of many authors and use ``$k$-means'' also to refer to a particular algorithm (as opposed to an optimization problem), as discussed below.

\section{Overview of Algorithms} \label{sec:kmeans:Algorithms}

% Explain the algorithm at a high-level -- do not talk about specific variations or implementation details. Give some theoretical background: Is the problem hard? What results can we expect?
The $k$-means problem is NP-hard in general Euclidean space (even for just two clusters) \cite{ADH09a} and for a general number of clusters (even in the plane) \cite{MNV10a}. However, the local-search heuristic proposed by \citeauthor{L82a}~\cite{L82a} performs reasonably well in practice. In fact, it is so ubiquitous today that it is often referred to as the standard algorithm or even just the $k$-means algorithm. At a high level, it works as follows:

\begin{enumerate}
	\item Seeding phase: Find initial positions for $k$ centroids $c_1, \dots, c_k$.
	\item Assign each point $x_1, \dots, x_n$ to its closest centroid. \label{enum:kmeans_abstract_points}
	\item Move each centroid $c$ to a position that minimizes the sum of distances between $c$ and each point assigned to $c$. (Note that if ``distance'' refers to the squared Euclidean distance, then this position is the barycenter/mean of all points assigned to $c$.)
	\item If convergence has been reached, stop. Otherwise, goto \eqref{enum:kmeans_abstract_points}.
\end{enumerate}

Since the value of the objective function decreases in every step, and there are only finitely many clusterings, the algorithm is guaranteed to converge to a local minimum \cite[Section~16.4]{CS08a}. While it is known that there are instances for which \citeauthor{L82a}'s heuristic takes exponentially many steps~\cite{V09a}, it has been shown that the algorithm has polynomial smoothed complexity \cite{AMR09a}---thus giving some theoretical explanation for good results in practice. With a clever seeding technique, \citeauthor{L82a}'s heuristic is moreover $O(\log k)$-competitive \cite{AV07a}.


\subsection{Algorithm Variants}

% Give an overview and references to variations that exist for this algorithm.
\paragraph{Seeding}

The quality of $k$-means is highly influenced by the choice of the seeding \cite{AV07a}. The following is a non-exhaustive list of options:
\begin{enumerate}
	\item Manual: User-specified list of initial centroid positions.
	\item Uniformly at random: Choose the $k$ centroids uniformly at random among the point set
	\item \texttt{$k$-means++}: Perform seeding so that the objective function is minimized in expectation \cite{AV07a}
	\item Use a different clustering algorithm for determining the seeding \cite{MNU00a}
	\item Run $k$-means with multiple seedings and choose the clustering with lowest cost
\end{enumerate}

\paragraph{Repositioning}

Most $k$-means formulations in textbooks do not detail the case where a centroid has no points assigned to it. It is an easy observation that moving a stray centroid in this case can only decrease the objective function. This can be done in a simple way (move onto a random point) or more carefully (e.g., move so that the objective function is minimized in expectation).

\paragraph{Convergence Criterion}

There are several reasonable convergence criteria. E.g., stop when:
\begin{enumerate}
	\item The number of repositioned centroids is below a given threshold
	\item The change in the objective drops below a given threshold
	\item The maximum number of iterations has been reached
	\item See, e.g., \textcite[Section~16.4]{CS08a} for more options.
\end{enumerate}

\paragraph{Variable Number of Clusters}

The number of clusters $k$ could be determined by the seeding algorithm (instead of being a parameter) \cite{MNU00a}. Strictly speaking, however, the algorithm should not be called $k$-means in this case.


\section{Seeding Algorithms}

In the following, we describe the seeding methods to be implemented for MADlib.

\subsection{Uniform-at-random Sampling}

Uniform-at-random sampling just uses the algorithms described in Section~\ref{sec:SampingWOReplacement}.

\subsection[k-means++]{$k$-means++}

\texttt{$k$-means++} seeding \cite{AV07a} starts with a single centroid chosen randomly among the input points. It then iteratively chooses new centroids from the input points until there are a total of $k$ centroids. The probability for picking a particular point is proportional to its minimum distance to any existing centroid. Intuitively, \texttt{$k$-means++} favors seedings where centroids are spread out over the whole range of the input points, while at the same time not being too susceptible to outliers.

\subsubsection{Formal Description}

\begin{algorithm}[$k$-means++$(k, P, \dist, C)$] \label{alg:kmeans++}
\alginput{Number of desired centroids $k$, set $P$ of points in $\R^d$, metric $\dist$, \\
	set $C$ of initial centroids}
\algoutput{Set of centroids $C$}
\begin{algorithmic}[1]
	\If{$C = \emptyset$}
		\State $C \set \{ \text{initial centroid chosen uniformly at random from } P \}$ \label{alg:kmeanspp:firstCentroid}
	\EndIf
	\While{$|C| < k$} \label{alg:kmeans++:for}
		\State $C \set C \cup \{ \text{random $p \in P$ with probability proportional to }\min_{c \in C} \dist(p,c) \}$ \label{alg:kmeanspp:nextcentroid}
	\EndWhile
\end{algorithmic}
\end{algorithm}

\begin{description}
	\item[Runtime] A naive implementation needs $\Theta(k^2 n)$ distance calculations, where $n = |P|$. A single distance calculation takes $O(d)$ time.
	\item[Space] Store $k$ centroids.
	\item[Subproblems]
		The existing \ref{sym:weighted_sample} subroutine can be used for:
		\begin{itemize}
			\item Line~\ref{alg:kmeanspp:firstCentroid}: Sample uniformly at random
			\item Line~\ref{alg:kmeanspp:nextcentroid}: Sample according to a discrete probability distribution.
		\end{itemize}
\end{description}

The number of distance calculations could be reduced by a factor of $k$ if we store, for each point $p \in P$, the distance to its closest centroid. Then, each iteration only needs $n$ distance calculations (i.e., only between the most recently added centroid and all points). In total, these are $\Theta(k n)$ distance calculations. Making this idea explicit leads to the following algorithm.

\begin{algorithm}[$k$-means++-ext$(k, P, \dist)$] \label{alg:kmeans++ext}
\alginput{Number of desired centroids $k$, set $P$ of points in $\R^d$, metric $\dist$, \\
	set $C$ of initial centroids}
\algoutput{Set of centroids $C$}
\algprecond{For all $p \in P: \delta[p] = \min_{c \in C} \dist(p, c)$ (or $\delta[p] = \infty$ if $C = \emptyset$)}
\begin{algorithmic}[1]
	\While{$|C| < k$} \label{alg:kmeans++ext:for}
		\State $\mathit{lastCentroid} \set \ref{sym:weighted_sample}(P, \delta)$ \label{alg:kmeans++ext:nextCentroid} \Comment{$\delta$ denotes the mapping $p \mapsto \delta[p]$}
		\State $C \set C \cup \{ \mathit{lastCentroid} \}$
		\For{$p \in P$} \label{alg:kmeans++ext:pointLoop}
			\If{$\dist(p, \mathit{lastCentroid}) < \delta[p]$}
				\State $\delta[p] \set \dist(p, \mathit{lastCentroid})$
			\EndIf
		\EndFor
	\EndWhile
\end{algorithmic}
\end{algorithm}

\begin{description}
	\item[Tuning] \label{kmeans++ext:tuning} The inner for-loop in line~\ref{alg:kmeans++ext:pointLoop} and \ref{sym:weighted_sample} in line~\ref{alg:kmeans++ext:nextCentroid} could be combined. With this improvement, only one pass over $P$ is necessary.
	\item[Runtime] $O(dkn)$ as explained before.
	\item[Space] Store $k$ centroids and $n$ distances.
	\item[Scalability] The outer while-loop is inherently sequential because the random variates in each iteration depend on all previous iterations. The inner loop, however, can be executed with data parallelism.
\end{description}

\subsubsection{Implementation as User-Defined Function}

In general, the performance benefit of explicitly storing points (i.e., choosing \ref{alg:kmeans++ext} over \ref{alg:kmeans++}) depends on the DBMS, the data, and the operating environment. The pattern of updating temporary state is made a bit more awkward in PostgreSQL due to its legacy of versioned storage. PostgreSQL performs an update by first inserting a new row and then marking the old row as invisible \cite[Section~23.1.2]{postgres:9.1.3}. As a result, for updates that touch many rows it is typically faster to copy the updated data into a new table (i.e., \texttt{CREATE TABLE AS SELECT} and \texttt{DROP TABLE}) rather than issue an \texttt{UPDATE}. Given these constraints, we currently choose to only implement algorithm~\ref{alg:kmeans++} (but not \ref{alg:kmeans++ext}) as the user-defined function \symlabel{kmeanspp\_seeding}{sym:kmeans++}.

\paragraph{In- and Output} The UDF expects the following arguments, and returns the following values:

\begin{center}
	\begin{tabularx}{\textwidth}{rlXl}
		\toprule%
		\textbf{Name} & \textbf{Description} & \textbf{Type}
		\\\otoprule
		In &
		\texttt{rel\_source} &
		Relation containing the points as rows &
		relation
		\\\midrule
		In &
		\texttt{expr\_point} &
		Point coordinates, i.e., the point $p$ &
		\specialcell{l}{expression\\ (floating-point vector)}
		\\\midrule
		In &
		\texttt{k} &
		Number of centroids &
		integer
		\\\midrule
		In &
		\texttt{fn\_dist} &
		Function returning the distance between two vectors &
		function
		\\\midrule
		In &
		\texttt{initial\_centroids} &
		Matrix containing the initial centroids as columns. This argument may be omitted (corresponding to an empty set $C$ of initial centroids). &
		floating-point matrix
		\\\midrule
		Out &
		&
		Matrix containing the $k$ centroids as columns &
		floating-point matrix
		\\\bottomrule
	\end{tabularx}
\end{center}

\paragraph{Components} The set of centroids $C$ is stored as a dense floating-point matrix that contains the centroids as columns vectors. Algoritm~\ref{alg:kmeans++} can be (roughly) translated into SQL as follow. We assume here that all function arguments are available as constants, and the matrix containing the centroids as columns is available as \texttt{centroids}. Line~\ref{alg:kmeanspp:firstCentroid} becomes:
\begin{lstlisting}[language=SQL,gobble=4]
    SELECT ARRAY[weighted_sample($expr_point, 1)]
    FROM $rel_source
\end{lstlisting}
Line~\ref{alg:kmeanspp:nextcentroid} is implemented using essentially the following SQL.
\begin{sql}[emph={centroids,fn_dist}]
    SELECT centroids || weighted_sample(
        $expr_point, (
            closest_column(
                centroids,
                $expr_point,
                fn_dist
        )).distance
    ) FROM $rel_source
\end{sql}
See Section~\ref{sec:matrix:closestColumn} for a description of \ref{sym:closestColumn}.

\subsubsection{Historical Implementations}

Implementation details and big-data heuristics that were used in previous versions of MADlib are documented here for completeness.

\begin{description}
	\item[v0.2.1beta and earlier] In lines~\ref{alg:kmeanspp:firstCentroid} and \ref{alg:kmeanspp:nextcentroid} of Algorithm~\ref{alg:kmeans++} use a random sample $P' \subsetneq P$.

		Here $P'$ will be a new random sample in each iteration. Under the a-priori assumption that a random point belongs to any of the $k$ (unknown) clusters with equal probability, sample enough points so that with high probability (e.g., $p = 0.999$) there is a point from each of the $k$ clusters.

		This is the classical occupancy problem (also called balls-into-bins model) \cite{F68a}: Throwing $r$ balls into $k$ bins, what is the probability that no bin is empty? The exact value is
		\begin{align*}
			u(r, k) = k^{-r} \sum_{i=0}^k (-1)^i \binom ki (k - i)^r
			\SiM.
		\end{align*}

		For $r,k \to \infty$ so that $r/k = O(1)$ we have the limiting form $u(r,k) \to (1 - e^{-r/k})^k =: \widetilde u(r, k)$. Rearranging $\widetilde u(r, k) > p$ gives $-\log(1 - \sqrt[k]p) \cdot k < r$. The smallest $r$ satisfying this inequality is chosen as the size of the sample set.
\end{description}

\section[Standard algorithm for k-means clustering]{Standard algorithm for $k$-means clustering}

The standard algorithm has been outlined in Section~\ref{sec:kmeans:Algorithms}. The formal description and our implementation are given below.

\subsection{Formal Description}

\begin{algorithm}[$k$-means$(k, P, \dist)$] \label{alg:kmeans}
\alginput{Set of initial centroids $C$, set $P$ of points, seeding strategy $\mathit{Seeding}$, metric $\dist$, \\
centroid function $\mathit{centroid}$, convergence strategy $\mathit{Convergence}$}
\algoutput{Set $C$ of final means}
\algprecond{$i = 0$}
\begin{algorithmic}[1]
	\Repeat
		\State $i \set i + 1$
		\State $C_\text{old} \set C$
		\State $C \set \bigcup_{c \in C} \{ \mathit{centroid} \{p \in P \mid \arg\min_{c' \in C} \dist(p, c') = c \} \}$ \label{alg:kmeans:MoveCentroids}
		\State $C \set \mathit{Seeding}(k, P, \dist, C)$ \label{alg:kmeans:Reseed} \Comment{Reseed ``lost'' centroids (if any)}
	\Until{$Convergence(C, C_\text{old}, P, i)$} \label{alg:kmeans:ConvergenceCond}
\end{algorithmic}
\end{algorithm}

\begin{description}
	\item[Runtime] See discussion in Section~\ref{sec:kmeans:Algorithms}.
	\item[Space] Store $2k$ centroids (both sets $C$ and $C_\text{old}$)
	\item[Scalability] The outer loop is inherently sequential. The recentering in line~\ref{alg:kmeans:MoveCentroids} is data-parallel (provided that the sets $C$ and $C_\text{old}$ are available on all computation nodes). Likewise, the convergence check in line~\ref{alg:kmeans:ConvergenceCond} is data-parallel if it only relies on distances between points $p$ and the set of centroids $C$, or the number of iterations.
\end{description}

\subsection{Implementation as User-Defined Function}

Algorithm~\ref{alg:kmeans} is implemented as the user-defined function \symlabel{kmeans}{sym:kmeans}. We choose to not make the convergence criterion a function argument but instead settle for parameters for the most typical criteria. Should the need arise, we might revoke that decision in the future. Moreover, the seeding strategy is currently not an argument, but \ref{sym:kmeans++} is always used in line~\ref{alg:kmeans:Reseed}.

\paragraph{In- and Output} The UDF expects the following arguments, and returns the following values:

\begin{center}
	\begin{tabularx}{\linewidth}{rlXl}
		\toprule%
		\textbf{Name} & \textbf{Description} & \textbf{Type}
		\\\otoprule
		In &
		\texttt{rel\_source} &
		Relation containing the points as tuples &
		relation
		\\\midrule
		In &
		\texttt{expr\_point} &
		Point coordinates, i.e., the point $p$ &
		\specialcell{l}{expression\\ (floating-point vector)}
		\\\midrule
		In &
		\texttt{initial\_centroids} &
		Matrix containing the initial centroids as columns &
		floating-point matrix
		\\\midrule
		In &
		\texttt{fn\_dist} &
		Function returning the distance between two vectors &
		function
		\\\midrule
		In &
		\texttt{agg\_centroid} &
		Aggregate returning the centroid for a set of points &
		function
		\\\midrule
		In &
		\texttt{max\_num\_iterations} &
		Convergence criterion: Maximum number of iterations &
		integer
		\\\midrule
		In &
		\texttt{min\_frac\_reassigned} &
		Convergence criterion: Convergence is reached if the fraction of points being reassigned to another centroid drops below \texttt{conv\_level} &
		floating-point
		\\\midrule
		Out & &
		Matrix containing the $k$ centroids as columns &
		floating-point matrix
		\\\bottomrule
	\end{tabularx}
\end{center}

\paragraph{Components}

The set of centroids $C$ is stored as a dense floating-point matrix that contains the centroids as columns vectors. Algoritm~\ref{alg:kmeans} can be (roughly) translated into SQL as follow. We assume here that all function arguments are available as constants, and the matrix containing the centroids is available as \texttt{centroids}. (These variables, which are unbound in the SQL statement, are shown in italic font.) Line~\ref{alg:kmeans:MoveCentroids} becomes:
%
\begin{sql}[emph={centroids,fn_dist}]
    SELECT matrix_agg(_centroid)
    FROM (
        SELECT $agg_centroid(_point) AS _centroid
        FROM (
            SELECT
                $expr_point AS _point,
                (closest_column(
                    centroids,
                    $expr_point,
                    fn_dist
                )).column_id AS _new_centroid_id
            FROM $rel_source
        ) AS _points_with_assignments
        GROUP BY _new_centroid_id
    ) AS _new_centroids
\end{sql}
See Section~\ref{sec:matrix:matrixAgg} for a description of \ref{sym:matrixAgg}.

It is a good idea to also compute the number of reassigned centroids, so that both line~\ref{alg:kmeans:MoveCentroids} and the convergence check in line~\ref{alg:kmeans:ConvergenceCond} can be computed with one pass over the data. To that end, we extend the inner-most query to also compute the previous closest centroid (i.e., we do a second \ref{sym:closestColumn} call where we pass matrix \texttt{old\_centroids} as first argument). During the aggregations in the two outer queries, we can then count (or sum up, respectively) the number of points that have been reassigned.

A caveat during testing whether a point has been reassigned is that centroid IDs are not constant over iterations: \ref{sym:closestColumn} returns a column index in the matrix \texttt{centroids}, and this matrix is the result of the \ref{sym:matrixAgg} aggregate---hence, the order of the columns is non-deterministic. We therefore cannot directly compare a column index from iteration $i$ to a column index from iteration $i - 1$, but instead need to translate the ``new'' index into an ``old'' index first. In order to do that, we extend the outermost query and also build up an array \texttt{old\_centroid\_ids}, where position $i$ will contain the column index that centroid $i$ had in the previous iteration. \Warning{A crucial assumption on the DBMS backend here is that the two aggregates \texttt{array\_agg} and \ref{sym:matrixAgg} see all tuples in the same order.} Putting everything together, the query becomes:
%
\begin{sql}[emph={centroids,old_centroids,old_centroid_ids,fn_dist}]
    SELECT
        matrix_agg(_centroid),        -- New value for: centroids
        array_agg(_new_centroid_id),  -- New value for: old_centroid_ids
        sum(_objective_fn),           -- New value for: objective_fn
        CAST(sum(_num_reassigned) AS DOUBLE PRECISION) / sum(_num_points)
                                      -- New value for: frac_reassigned
    FROM (
        SELECT
            (_new_centroid).column_id AS _new_centroid_id,
            sum((_new_centroid).distance) AS _objective_fn,
            count(*) AS _num_points,
            sum(
                CAST(
                    old_centroid_ids[(_new_centroid).column_id + 1] != _old_centroid_id
                    AS INTEGER
                )
            ) AS _num_reassigned,
            $agg_centroid(_point) AS _centroid
        FROM (
            SELECT
                $expr_point AS _point,
                closest_column(
                    centroids,
                    $expr_point,
                    fn_dist
                ) AS _new_centroid,
                (closest_column(
                    old_centroids,
                    $expr_point,
                    fn_dist
                )).column_id AS _old_centroid_id
            FROM $rel_source
        ) AS _points_with_assignments
        GROUP BY (_new_centroid).column_id
    ) AS _new_centroids
\end{sql}


Finally, line~\ref{alg:kmeans:Reseed} is simply implemented by calling function \ref{sym:kmeans++}. For a slight performance benefit, this function call should be guarded by a check if the number of centroids is lower than $k$.


% When using TeXShop on the Mac, let it know the root document.
% The following must be one of the first 20 lines.
% !TEX root = ../design.tex

\chapter{Low-rank Matrix Factorization}

% Abstract. What is the problem we want to solve?
This module implements "factor model" for representing a sparse matrix
using a low-rank approximation \cite{DBLP:conf/icml/SrebroJ03}.
Mathematically, this model seeks to find matrices U and V (also referred 
as factors) that, for any given sparse matrix A, minimizes:
\[ ||\boldsymbol A - \boldsymbol UV^{T} ||_2 \]
subject to $rank(\boldsymbol UV^{T}) \leq r$, where $||\cdot||_2$ denotes
the Frobenius norm and $1 \leq r \leq rank(\boldsymbol A)$.
Let $A$ be a $m \times n$ matrix, then $U$ will be $m \times r$ and $V$
will be $n \times r$, in dimension.
This model is not intended to do the full decomposition, or to be used as
part of inverse procedure.
This model has been widely used in recommendation systems, such as Netflix
\cite{:TheNetflixPrize07}, and image processing
\cite{DBLP:conf/nips/WrightGRPM09}.

\section{Incremental Gradient Descent}

% Background. Why can we solve the problem with incremental gradient?
\subsection{Solving as a Convex Program}
Recent work
\cite{DBLP:journals/cacm/CandesR12, DBLP:journals/siamrev/RechtFP10}
has demonstrated that the low-rank matrix factorization can be solved as
a convex programming problem.
This body of work enables us to solve the problem by using gradient-based
line search algorithms.
Among many of these algorithms, incremental gradient descent algorithm is
a popular choice, especially for really large input matrices
\cite{DBLP:conf/sigmod/FengKRR12, DBLP:conf/kdd/GemullaNHS11}.

\subsection{Formal Description}
\begin{algorithm}[lmf-igd$(r, A, \alpha)$] \label{alg:lmf-igd}
\alginput{Sparse matrix $A$,
\\step size $\alpha$,
\\low-rank constraint $r$, 
\\convergence strategy $\mathit{Convergence}$,
\\random factor generator $\mathit{GenerateFactor}$}
\algoutput{Factors $U$ ($m \times r$) and $V$ ($n \times r$)}
\algprecond{$iteration = 0$}
\begin{algorithmic}[1]
	\State $U \set GenerateFactor(m, r)$
	\State $V \set GenerateFactor(n, r)$
	\Repeat
		\State $iteration \set iteration + 1$
		\State $U_\text{old} \set U$
		\State $V_\text{old} \set V$
		\For{$(i, j, y) \in A$} \Comment{Single entry in sparse matrix A}
			\State $e \set U_i \cdot V_j - y$
			\State $temp \set U_i - \alpha e V_j$
			\State $V_j \set V_j - \alpha e U_i$ \Comment{In-place update of V}
			\State $U_i \set temp$ \Comment{In-place update of U}
		\EndFor
	\Until{$Convergence(U_\text{old}, V_\text{old}, U, V, iteration)$}
\end{algorithmic}
\end{algorithm}

\begin{description}
	\item[Runtime] $O(N_{A} (m + n) r + m n r)$ for one iteration,
        where $N_{A}$ is the number of nonzero elements in $A$.

	\item[Space] Store the $temp$, a $r$-floating-point vector.

	\item[Scalability] The outer loop is inherently sequential.
        The inner loop is data-parallel and model averaging 
        \cite{DBLP:conf/nips/DuchiAW10} is used.

    \item[Factor Initialization] The author of this document is not aware that
        significant differences are caused if random factors are initialized
        by different distributions. But zero values should be avoided. And
        entries in factors should not be initialized as the same value;
        otherwise, factors will always be rank 1.

    \item[Convergence Criterion] Usually, following conditions are combined by
        AND, OR, or NOT.
        \begin{enumerate}
            \item The change in the objective drops below a given threshold
                (E.g., RMSE)
            \item The value of the objective matches some pre-computed value
            \item The maximum number of iterations has been reached
            \item There could be more.
        \end{enumerate}
\end{description}

% When using TeXShop on the Mac, let it know the root document.
% The following must be one of the first 20 lines.
% !TEX root = ../design.tex

\chapter[Convex Programming Framework]{Convex Programming Framework}
% Motivation. Why do we want to have this abstract layer?
The nature of MADlib drives itself to support many different kinds of data modeling modules, such as logistic regression, support vector machine, matrix factorization, etc.
However, keeping up with the state of the art and experimenting with individual data modeling modules require significant development and quality-assurance effort.
Therefore, to lower the bar of adding and maintaining new modules, it is crucial to identify the invariants among many important modules, in turn, abstract and encapsulate them as reusable components.

Bismarck \cite{DBLP:conf/sigmod/FengKRR12} is such a unified framework that links many useful statistical modeling modules and the relational DBMS, by introducing a well-studied formulation, convex programming, in between.
Incremental Gradient Descent (IGD) has also been shown as a very effective algorithm to solve convex programs in the relational DBMS environment.
But it is natural that IGD does not always fit the need of MADlib users who are applying convex statistical modeling to various domains.
Driven by this, convex programming framework in MADlib also implements other algorithms that solves convex programs, such as Newton's method and conjugate gradient methods.

\section{Introduction}
% Problem definition. What are the problems that we can solve, formally and example applications?
% linearly separable, unconstrained, continuous, deterministic, convex, minimization problems.
This section is to first explain, formally, the type of problems that we consider in the MADlib convex programming framework, and then give a few example modules.

\subsection{Formulation}
We support numerical optimization problems with an objective function that is a sum of many component functions \cite{springerlink:10.1007/s10107-011-0472-0}, such as
\[\min_{w \in \mathbb{R}^N} \sum_{m=1}^M f_{z_m}(w),\]
where $z_m \in \mathcal{O}, m = 1,...,M$, are observations, and $f_{z_m} : \mathbb{R}^N \to \mathbb{R}$ are convex functions.
For simplicity, let $z_{1:M}$ denote $\{z_m \in \mathcal{O} | m = 1,...,M\}$.
Note: given $z_{1:M}$, let $F(w) = \sum_{m=1}^M f_{z_m}(w)$, and $F : \mathbb{R}^N \to \mathbb{R}$ is also convex.

\subsection{Examples}
Many popular models can be formulated in the above form, with $f_{z_m}$ being properly specified.

\paragraph{Logistic Regression.} The component function is given by
\[f_{(x_m, y_m)}(w) = \log(1 + e^{- y_m w^{T} x_m}),\]
where $x_m \in \mathbb{R}^N$ are values of independent variables, and $y_m \in \{-1, 1\}$ are values of the dependent variable, $m = 1,...,M$.

\paragraph{Linear SVM with hinge loss.} The component function is given by
\[f_{(x_m, y_m)}(w) = \max(0, 1 - y_m w^{T} x_m),\]
where $x_m \in \mathbb{R}^N$ are values of features, and $y_m \in \{-1, 1\}$ are values of the label, $m = 1,...,M$.
Bertsekas \cite{springerlink:10.1007/s10107-011-0472-0} gives many other examples across application domains.

\section{Algorithms}
\paragraph{Gradient Descent.}
A most-frequently-mentioned algorithm that solves convex programs is gradient descent.
This is an iterative algorithm and the iteration is given by
\[w_{k+1} = w_k - \alpha_k \nabla F(w_k),\]
where, given $z_{1:M}$, $F(w) = \sum_{m=1}^M f_{z_m}(w)$, and $\alpha_k$ is a positive scalar, called stepsize (or step length).
Gradient descent algorithm is simple but usually recognized as a slow algorithm with linear convergence rate, while other algorithms like conjugate gradient methods and Newton's method has super-linear convergence rates \cite{nocedal2006numerical}.

\paragraph{Line Search: A Class of Algorithms.}
% line search & trust region
Convex programming has been well studied in the past few decades, and two main classes of algorithms are widely considered: line search and trust region (\cite{nocedal2006numerical}, section 2.2).
Because line search is more commonly deployed and discussed, we focus on line search in MADlib, although some of the algorithms we discuss in this section can also easily be formulated as trust region strategy.
% general form of line search: w += \alpha p_k
All algorithms of line search strategies have the iteration given by
\[w_{k+1} = w_k + \alpha_k p_k,\]
where $p_k \in \mathbb{R}^N$ is search direction, and stepsize $\alpha_k$ \cite{nocedal2006numerical}.
Specifiedly, for gradient descent, $p_k$ is the steepest descent direction $- \nabla \sum_{m=1}^M f_{z_m}(w_k)$.

\subsection{Formal Description of Line Search}
\begin{algorithm}[line-search$(z_{1:M})$] \label{alg:line-search}
\alginput{Observation set $z_{1:M}$,\\
convergence criterion $\mathit{Convergence}()$,\\
start strategy $\mathit{Start}()$,\\
initialization strategy $\mathit{Initialization}()$,\\
transition strategy $\mathit{Transition}()$,\\
finalization strategy $\mathit{Finalization}()$}
\algoutput{Coefficients $w \in \mathbb{R}^N$}
\algprecond{$iteration = 0, k = 0$}
\begin{algorithmic}[1]
	\State $w_\text{new} \set \mathit{Start}(z_{1:M})$
	\Repeat
        \State $w_\text{old} \set w_\text{new}$
        \State $\mathit{state} \set \mathit{Initialization}(w_\text{new})$
		\For{$m \in 1..M$} \Comment{Single entry in the observation set}
			\State $\mathit{state} \set \mathit{Transition}(\mathit{state}, z_m)$
                \Comment{Usually computing derivative}
		\EndFor
		\State $w_\text{new} \set Finalization(\mathit{state})$
	\Until{$Convergence(w_\text{old}, w_\text{new}, \mathit{iteration})$}
    \State \Return $w_\text{new}$
\end{algorithmic}
\end{algorithm}

\paragraph{Programming Model.}
We above give the algorithm of generic line search strategy, in the fashion of the selected programming model supported by MADlib (mainly user-defined aggregate).

\paragraph{Parallelism.}
The outer loop is inherently sequential.
We require the inner loop is data-parallel.
Simple component-wise addition or model averaging \cite{DBLP:conf/nips/DuchiAW10} are used to merge two states.
A merge function is not explicitly added to the pseudocode for simplicity.
A separate discussion will be made when necessary.

\paragraph{Convergence criterion.}
Usually, following conditions are combined by AND, OR, or NOT.
\begin{enumerate}
    \item The change in the objective drops below a given threshold (E.g., negative log-likelihood, root-mean-square error).
    \item The value of the objective matches some pre-computed value.
    \item The maximum number of iterations has been reached.
    \item There could be more.
\end{enumerate}
In MADlib implementation, the computation of objective is paired up with line-search to share data I/O.

\paragraph{Start strategy.}
In most cases, zeros are used unless otherwise specified.

\paragraph{Transition and finalization strategies.}
The coefficients update code ($w_{k+1} \set w_k + \alpha_k p_k$) is put into either $\mathit{Transition}()$ or $\mathit{Finalization}()$.
These two functions contain most of the computation logic, for computing the search direction $p_k$.
We discuss details of individual algorithms in the following sections.
For simplicity, global iterator $k$ is read and updated in place by these functions without specifed as an additional argument.

\subsection{Incremental Gradient Descent (IGD)}
% motivation and introduction of IGD
A main challenge arises when we are handling large amount of data, $M \gg 1$, where the computation of $\nabla (\sum_{m=1}^M f_{z_m})$ requires a whole pass of the observation data which is usually expensive.
What distinguishes IGD from other algorithms is that it approximates  $\nabla (\sum_{m=1}^M f_{z_m}) = \sum_{m=1}^M (\nabla f_{z_m})$ by the gradient of a single component function $\nabla f_{z_m}$
\footnote{$z_m$ is usually selected in a stochastic fashion.
Therefore, IGD is also referred to as stochastic gradient descent.
The convergence and convergence rate of IGD are well developed \cite{springerlink:10.1007/s10107-011-0472-0}, and IGD is often considered to be very effective with $M$ being very large \cite{DBLP:conf/nips/BottouB07}.}.
The reflection of this to the pseudocode makes the coefficients update code ($w_{k+1} \set w_k + \alpha_k p_k$) in $\mathit{Transition}()$ instead of in $\mathit{Finalization}()$.

\subsubsection{Initialization Strategy}
\begin{algorithm}[initialization-igd$(w)$] \label{alg:initialization-igd}
\alginput{Coefficients $w \in \mathbb{R}^N$}
\algoutput{Transition state $\mathit{state}$}
\begin{algorithmic}[1]
    \State $\mathit{state}.w_k \set w$
    \State \Return $\mathit{state}$
\end{algorithmic}
\end{algorithm}

\subsubsection{Transition Strategy}
\begin{algorithm}[transition-igd$(\mathit{state}, z_m)$] \label{alg:transition-igd}
\alginput{Transition state $\mathit{state}$,\\
observation entry $z_m$,\\
stepsize $\alpha \in \mathbb{R}^+$,\\
gradient function $\mathit{Gradient}()$}
\algoutput{Transition state $\mathit{state}$}
\begin{algorithmic}[1]
    \State $p_k \set - \mathit{Gradient}(\mathit{state}.w_k, z_m)$
        \Comment{Previously mentioned as $p_k = - \nabla f_{z_m}$}
    \State $\mathit{state}.w_{k+1} \set \mathit{state}.w_k + \alpha p_k$
    \State $k \set k + 1$
        \Comment{In-place update of the global iterator}
    \State \Return $\mathit{state}$
\end{algorithmic}
\end{algorithm}

\paragraph{Stepsize.}
In MADlib, we support only constant stepsize for simplicity.
Although IGD with constant stepsizes does not even have convergence guarantee \cite{springerlink:10.1007/s10107-011-0472-0}, but it works reasonably well in practice so far \cite{DBLP:conf/sigmod/FengKRR12} with some proper tuning.
Commonly-used algorithms to tune stepsize (\cite{bertsekas1999nonlinear}, appendix C) are mostly heuristics and do not have strong guarantees on convergence rate.
More importantly, these algorithms require many evaluations of the objective function, which is usually very costly in use cases of MADlib.

\paragraph{Gradient function.}
A function where IGD accepts computational logic of specified modules.
In MADlib convex programming framework, currently, there is no support of objective functions that does not have gradient or subgradient.
Those objective functions that is not linearly separable is not currently supported by the convex programming framework, such as Cox proportional hazards models \cite{Cox1972}.

\subsubsection{Finalization Strategy}
\begin{algorithm}[finalization-igd$(\mathit{state})$] \label{alg:finalization-igd}
\alginput{Transition state $\mathit{state}$}
\algoutput{Coefficients $w \in \mathbb{R}^N$}
\begin{algorithmic}[1]
    \State \Return $\mathit{state}.w_k$
        \Comment{Trivially return $w_k$}
\end{algorithmic}
\end{algorithm}

\subsection{Conjugate Gradient Methods}
% Simple description of conjugate gradient
Conjugate gradient methods that solve convex programs are usually refered to as nonlinear conjugate gradient mthods.
The key difference between conjugate gradient methods and gradient descent is that conjuagte gradient methods perform adjustment of the search direction $p_k$ by considering gradient directions of previous iterations in some intriguing way.
We skip the formal desciption of conjugate gradient methods that can be found in the references (such as Nocedal \& Wright \cite{nocedal2006numerical}, section 5.2). 

\subsubsection{Initialization Strategy}
\begin{algorithm}[initialization-cg$(w)$] \label{alg:initialization-cg}
\alginput{Coefficients $w \in \mathbb{R}^N$,\\
gradient value $g \in \mathbb{R}^N$ (i.e., $\sum_{m=1}^M \nabla f_{z_m}(w_{k-1})$),\\
previous search direction $p \in \mathbb{R}^N$}
\algoutput{Transition state $\mathit{state}$}
\begin{algorithmic}[1]
    \State $\mathit{state}.p_{k-1} \set p$
    \State $\mathit{state}.g_{k-1} \set g$
    \State $\mathit{state}.w_k \set w$
    \State $\mathit{state}.g_k \set 0$
    \State \Return $\mathit{state}$
\end{algorithmic}
\end{algorithm}

\subsubsection{Transition Strategy}
\begin{algorithm}[transition-cg$(\mathit{state}, z_m)$] \label{alg:transition-cg}
\alginput{Transition state $\mathit{state}$,\\
observation entry $z_m$,\\
gradient function $\mathit{Gradient}()$}
\algoutput{Transition state $\mathit{state}$}
\begin{algorithmic}[1]
    \State $\mathit{state}.g_k \set \mathit{state}.g_k + \mathit{Gradient}(\mathit{state}.w_k, z_m)$
    \State \Return $\mathit{state}$
\end{algorithmic}
\end{algorithm}

\subsubsection{Finalization Strategy}
\begin{algorithm}[finalization-cg$(\mathit{state})$] \label{alg:finalization-cg}
\alginput{Transition state $\mathit{state}$,\\
stepsize $\alpha \in \mathbb{R}^+$,\\
update parameter strategy $\mathit{Beta}()$}
\algoutput{Coefficients $w \in \mathbb{R}^N$,\\
gradient value $g \in \mathbb{R}^N$ (i.e., $\sum_{m=1}^M \nabla f_{z_m}(w_{k-1})$),\\
previous search direction $p \in \mathbb{R}^N$}
\begin{algorithmic}[1]
    \If{k = 0}
        \State $\mathit{state}.p_k \set - \mathit{state}.g_k$
    \Else
        \State $\beta_k \set \mathit{Beta}(state)$
        \State $p_k \set - \mathit{state}.g_k + \beta_k p_{k-1}$
    \EndIf
    \State $\mathit{state}.w_{k+1} \set \mathit{state}.w_k + \alpha p_k$
    \State $k \set k + 1$
    \State $p \set p_{k-1}$
        \Comment{Implicitly returning}
    \State $g \set \mathit{state}.g_{k-1}$
        \Comment{Implicitly returning again}
    \State \Return $\mathit{state}.w_k$
\end{algorithmic}
\end{algorithm}

\paragraph{Update parameter strategy.}
For cases that $F$ is strongly convex quadratic (e.g., ordinary least squares), $\beta_k$ can be computed in closed form, having $p_k$ be in conjugate direction of $p_0,...,p_{k-1}$.
For more general objective functions, many different choices of update parameter are proposed \cite{hager2006survey, nocedal2006numerical}, such as
\[\beta_k^{FR} = \frac{\|g_k\|^2}{\|g_{k-1}\|^2},\]
\[\beta_k^{HS} = \frac{g_k^T (g_k - g_{k-1})}{p_{k-1}^T (g_k - g_{k-1})},\]
\[\beta_k^{PR} = \frac{g_k^T (g_k - g_{k-1})}{\|g_{k-1}\|^2},\]
\[\beta_k^{DY} = \frac{\|g_k\|^2}{p_{k-1}^T (g_k - g_{k-1})},\]
where $g_k = \sum_{m=1}^M \nabla f_{z_m}(w_k)$, and $p_k = - g_k + \beta_k p_{k-1}$.
We choose the strategy proposed by Dai and Yuan due to lack of mechanism for stepsize tuning in MADlib, which is required for other alternatives to guarantee convergence rate. (See Theorem 4.1 in Hager and Zhang \cite{hager2006survey}).
In fact, lack of sufficient stepsize tuning for each iteration individually could make conjugate gradient methods have similar or even worse convergence rate than gradient descent.
This should be fixed in the future.

\subsection{Newton's Method}
Newton's method uses a search direction other than the steepest descent direction -- \emph{Newton direction}.
The Newton direction is very effective in the cases that the objective function is not too different from a quadratic approximation, and it gives quadratic convergence rate by considering Taylor's theorem.
Formally, the Newton direction is given by
\[p_k = -(\nabla^2 F(w_k))^{-1} \nabla F(w_k),\]
where, given $z_{1:M}$, $F(w) = \sum_{m=1}^M f_{z_m}(w)$, and $H_k = \nabla^2 F(w_k)$ is called the Hessian matrix.

\subsubsection{Initialization Strategy}
\begin{algorithm}[initialization-newton$(w)$] \label{alg:initialization-newton}
\alginput{Coefficients $w \in \mathbb{R}^N$}
\algoutput{Transition state $\mathit{state}$}
\begin{algorithmic}[1]
    \State $\mathit{state}.w_k \set w$
    \State $\mathit{state}.g_k \set 0$
    \State $\mathit{state}.H_k \set 0$
    \State \Return $\mathit{state}$
\end{algorithmic}
\end{algorithm}

\subsubsection{Transition Strategy}
\begin{algorithm}[transition-newton$(\mathit{state}, z_m)$] \label{alg:transition-newton}
\alginput{Transition state $\mathit{state}$,\\
observation entry $z_m$,\\
gradient function $\mathit{Gradient}()$,\\
Hessian matrix function $\mathit{Hessian}()$}
\algoutput{Transition state $\mathit{state}$}
\begin{algorithmic}[1]
    \State $\mathit{state}.g_k \set \mathit{state}.g_k + \mathit{Gradient}(\mathit{state}.w_k, z_m)$
    \State $\mathit{state}.H_k \set \mathit{state}.H_k + \mathit{Hessian}(\mathit{state}.w_k, z_m)$
    \State \Return $\mathit{state}$
\end{algorithmic}
\end{algorithm}

\subsubsection{Finalization Strategy}
\begin{algorithm}[finalization-newton$(\mathit{state})$] \label{alg:finalization-newton}
\alginput{Transition state $\mathit{state}$}
\algoutput{Coefficients $w \in \mathbb{R}^N$}
\begin{algorithmic}[1]
    \State $p_k \set - (\mathit{state}.H_k)^{-1} \mathit{state}.g_k$
    \State $\mathit{state}.w_{k+1} \set \mathit{state}.w_k + p_k$
    \State $k \set k + 1$
    \State \Return $\mathit{state}.w_k$
\end{algorithmic}
\end{algorithm}

\paragraph{Hessian Matrix Function.}
A function where Newton's method accepts another computational logic of specified modules. See also gradient function.

\paragraph{Inverse of the Hessian Matrix.}
The inverse of Hessian matrix may not always exist if the Hessian is not guaranteed to be positive definite ($\nabla^2 F = 0$ when $F$ is linear).
We currently only support Newton's method for objetcive functions that is strongly convex.
This may sometimes mean an objective function that is not globally strongly convex but Newton's method works well with a good starting point as long as the objective function is strongly convex in a convex set that contains the given starting point and the minimum.
A few techniques that modify the Newton's method to adapt objective functions that are not strongly convex can be found in the references \cite{bertsekas1999nonlinear, nocedal2006numerical}.

\paragraph{Feed a Good Start Point.}
Since Newton's method is sensitive to the start point $w_0$, we provide a start strategy $\mathit{Start()}$ to accept a start point that may not be zeros.
It may come from results of other algorithms, e.g., IGD.

\subsection{Quasi-Newton Method}


% When using TeXShop on the Mac, let it know the root document.
% The following must be one of the first 20 lines.
% !TEX root = ../design.tex

\chapter[Linear-chain Conditional Random Field]{Linear-chain Conditional Random Field}
% Motivation. Why do we want to have this abstract layer?
Conditional random field(CRFs) is a type of discriminative undirected probabilistic graphical model.
Linear-chain CRFs are a special CRFs which assumes that the next state only depends on the current state. 
Linear-chain CRFs achieved the start of art accuracy in some real world natural language processing tasks such
as part of speech tagging(POS) and named entity resolution(NER).

\section{Linear-chain CRF Learning}

\subsection{Mathematical Notations}
\begin{itemize}
\item $p(\boldsymbol Y | \boldsymbol X)$: conditional probability distributions of label sequence $\boldsymbol Y$ given input sequence $\boldsymbol X$.
\item $M$: total number of unique features.
\item $I$: the positon of last token in a training sentence.
\item $N$: number of sentences in the training data set.
\item $\lambda$: the coefficient(feature weight).
\item $\ell_{\lambda}$: log-likelihood summarized over all training sentences.
\item $\nabla \ell_{\lambda}$: gradient vector summarized over all training sentences.
\item $\ell_{\lambda}^\prime$: adjusted log-likelihood to avoid over fitting using spherical Gaussian weight prior.
\item $\nabla \ell_{\lambda}^\prime$: adjusted gradient vector to avoid over fitting using spherical Gaussian weight prior.

\end{itemize}

\subsection{Formulation}
Conditional random fields(CRFs) is a type of discriminative undirected probabilistic graphical model.
A linear-chain CRF is a distribution
    \[p(\boldsymbol Y | \boldsymbol X) = \frac{\exp{\sum_{m=1}^M \sum_{i=0}^{I} \lambda_m f_m(y_i,y_{i-1},x_i)}}{Z(X)}\]

Where Z(X) is an instance specific normalizer
\[Z(X) = \sum_{y} \exp{\sum_{m=1}^M \sum_{i=0}^{I} \lambda_m f_m(y_i,y_{i-1},x_i)}\].

Train a CRF by maximizing the log-likelihood of a giving training set $ T=\{(x_k,y_k)\}_{k=1}^N$.
Seek the zero of the gradient.\\
    \[\ell_{\lambda}=\sum_k \log p_\lambda(y_k|x_k) =\sum_k[\lambda F(y_k,x_k)-\log Z_\lambda(x_k)]\]
    \[\nabla \ell_{\lambda}=\sum_k[\lambda F(y_k,x_k)-E_{p\lambda(Y|x_k)}F(Y,x_k)]\]

To avoid overfitting, we penalize the likelihood with a spherical Gaussian weight prior:\\
    \[\ell_{\lambda}^\prime=\sum_k[\lambda F(y_k,x_k)-\log Z_\lambda(x_k)]-\frac{\lVert \lambda \rVert^2}{2\sigma ^2}\]
    \[\nabla \ell_{\lambda}^\prime=\sum_k[\lambda F(y_k,x_k)-E_{p\lambda(Y|x_k)}F(Y,x_k)]-\frac{\lambda}{\sigma ^2}\]
Note:We hard code sigma to $100$ in the implementation which follows other CRF packages in the literature.

\subsection{Forward-backward Algorithm}
$E_{p\lambda(Y|x)}F(Y,x)$ is computed using a variant of the forward-backward algorithm:

    \[E_{p\lambda(Y|x)}F(Y,x) = \sum_y p\lambda(y|x)F(y,x) = \sum_i\frac{\alpha_{i-1}(f_i*M_i)\beta_i^T}{Z_\lambda(x)}\]
    \[$Z_\lambda(x) = \alpha_I.1^T$\]
    where $\alpha_i$ and $\lambda_i$ the forward and backward state cost vectors defined by\\
  \[\alpha_i = 
    \begin{cases}
    \alpha_{i-1}M_i, & 0<i<=n\\
    1, & i=0
    \end{cases}
    ,
    \beta_i^T = 
    \begin{cases}
    M_{i+1}\lambda_{i+1}^T, & 1<=i<I\\
    1, & i=n
    \end{cases}
  \]

\subsection{L-BFGS Convex Solver}
The limited-memory BFGS(L-BFGS) is the limited memory variation of the Broyden-Fletcher-Goldfarb-Shanno(BFGS).
It is the state of art of large scale non-constraint convex optimization method.
We translate a in-memory java implementation to a c++ in-database implementation with Eigen support.
Eigen vector and Eigen matrix are used instead of the plain one dimentional and two dimentional arrays.
In the java in-memeory implementation, it defines many static variables which are shared between the interations.
However, in the MADlib implementation, we define these variables in the state object.
Before one iteration of L-BFGS optimization, we need to initialize the L-BFGS with current the state object. 
At the end of one iteration, we need to dump the updated variables to the the database state for next iteration.


\subsection{Parallel CRF Training}
\begin{algorithm}[CRF training$(z_{1:M})$] \label{alg:CRF training}
\alginput{Observation set $z_{1:M}$,\\
convergence criterion $\mathit{Convergence}()$,\\
start strategy $\mathit{Start}()$,\\
initialization strategy $\mathit{Initialization}()$,\\
transition strategy $\mathit{Transition}()$,\\
finalization strategy $\mathit{Finalization}()$}
\algoutput{Coefficients $w \in \mathbb{R}^N$}
\algprecond{$iteration = 0, diag = \boldsymbol 1$}
\begin{algorithmic}[1]
	\State $w_\text{new} \set \mathit{Start}(z_{1:M})$
	\Repeat
        \State $w_\text{old} \set w_\text{new}$
        \State $\mathit{state} \set \mathit{Initialization}(w_\text{new})$
		\For{$m \in 1..M$} \Comment{Single entry in the observation set}
			\State $\mathit{state} \set \mathit{Transition}(\mathit{state}, z_m)$
                \Comment{Computing gradient and log-likelihood.}
		\EndFor
		\State $w_\text{new} \set Finalization(\mathit{state})$ \Comment{Mainly invoke L-BFGS convex solver}
	\Until{$Convergence(w_\text{new}, g_\text{new}, \mathit{iteration})$}
    \State \Return $w_\text{new}$
\end{algorithmic}
\end{algorithm}

\paragraph{Programming Model.}
We above give the algorithm of parallel CRF training strategy, in the fashion of the selected programming model supported by MADlib (mainly user-defined aggregate).

\paragraph{Parallelism.}
The outer loop is inherently sequential over multiple iterations.
The iteration $n+1$ takes the output of iteration $n$ as input, so on so forth until the stop criterion is satisfied.
The inner loop which calculates the gradient and log-likelihood for each document is data-parallel.
Simple model averaging are used to merge two states.
A merge function is not explicitly added to the pseudocode for simplicity.
The finalization function invokes the L-BFGS convex solver to get a new solution. L-BFGS is sequential, but very fast.
Experiments show that the speed-up ration approaches the number of segments configured in the Greenplum database.

\paragraph{Convergence criterion.}
Usually, following conditions are combined by AND, OR, or NOT.
\begin{enumerate}
    \item The norm of gradient divided by the norm of coefficient drops below a given threshold.
    \item The maximum number of iterations has been reached.
    \item There could be more.
\end{enumerate}

\paragraph{Start strategy.}
In most cases, zeros are used unless otherwise specified.

\paragraph{Transition strategies.}
This function contains the logic of computing the gradient and log-likelihood for each tuple using the forward-backward
algorithm. The algorithms will be discussed in the following sections.

\begin{algorithm}[transition-lbfgs$(\mathit{state}, z_m)$] \label{alg:transition-lbfgs}
\alginput{Transition state $\mathit{state}$,\\
observation entry $z_m$,\\
gradient function $\mathit{Gradient}()$}
\algoutput{Transition state $\mathit{state}$}
\begin{algorithmic}[1]
    \State $\{state.g,state.loglikelihood\}  \set \mathit{Gradient}(\mathit{state}, z_m)$
        \Comment{using forward-backward algorithm to calculate gradient and loglikelihood}
    \State $\mathit{state}.num\_rows \set \mathit{state}.num\_rows + 1$
    \State \Return $\mathit{state}$
\end{algorithmic}
\end{algorithm}


\paragraph{Merge strategies.}
The merge function simply sums the gradient and loglikelihood over all training documents 
\begin{algorithm}[merge-lbfgs$(\mathit{state_1}, \mathit{state_2})$] \label{alg:merge-lbfgs}
\alginput{Transition state $\mathit{state_1}$,\\
Transition state $\mathit{state_2}$}
\algoutput{Transition state $\mathit{state_{new}}$}
\begin{algorithmic}[1]
    \State $\mathit{state_{new}}.g \set \mathit{state_1}.g + \mathit{state_2}.g$
    \State $\mathit{state_{new}}.loglikelihood \set \mathit{state_1}.loglikelihood + \mathit{state_2}.loglikelihood$
    \State \Return $\mathit{state_{new}}$
\end{algorithmic}
\end{algorithm}


\paragraph{Finalization strategy.}
The finalization function invokes the L-BFGS convex solver to get a new coefficent vector.\\

\begin{algorithm}[finalization-lbfgs$(state)$] \label{alg:CRF training}
\alginput{Transition state $state$,\\
LBFGS $\mathit{lbfgs}()$}
\algoutput{Transition state  $state$}
\begin{algorithmic}[1]
        \State $\{state.g,state.loglikelihood\} \set penalty(state.g,state.loglikelihood)$ \Comment{To avoid overfitting, add penalization} 
        \State $\{state.g,state.loglikelihood\}\set-\{state.g,state.loglikelihood\}$ \Comment{negation for maximization}
        \State LBFGS instance($state)$ \Comment{initialize the L-BFGS instance with previous state}
        \State instance.$lbfgs()$ \Comment{invoke the L-BFGS convex solver}
        \State instance.save\_state$(state)$ \Comment{save updates variables to the state for next iteration}
        \State \Return $state$
\end{algorithmic}
\end{algorithm}

Feeding with current solution, gradient, log-likelihood, etc., the L-BFGS will ouput a new solution.
To avoid overfitting, a penalization function is needed. We choose to penalize the loglikelihood with a spherical Gaussian weight prior.
Also, L-BFGS is to maximum objective, so we need to negate the gradient vector and loglikelihood to fit your needs to minimize the loglikehood.

\section{Linear-chain CRF Applications}
Linear-chain CRF can be used in various applications such as  part of speech tagging and named entity resolution.
All the following section assumes that the application is part of speech tagging. It can be fitted to named entity resolution with minimal effort. 

\subsection{Part of Speech Tagging}
Part-of-speech tagging, also called grammatical tagging or word-category disambiguation, is the process of assigning 
a part of speech to each word in a sentence. POS has been widely used in information retrieval, text to speech. There are two distinct methods for 
POS task: rule-based and stochastic.
In rule-based method, it defines large collection of rules to indentify the tag. Stochastic method is based on probabilistic 
graphic models such as hidden markov models, conditional random fields. In practice, conditional random fields have been approved 
to achieve the state of art of accuracy.
\subsection{Tag Set}
There are various tag set used in the literature. The Pennsylvania Treebank tag-set is the commonly used tag set used. It contains
45 tags. The following list part of tags in the tag set.

\begin {table}[h]
\caption {Pen Treebank III Tag Set} \label{tab:title} 
\[\begin{tabular}{lll||lll}
  Tag & Description         & Example       & Tag & Description         & Example\\
  \hline                        
  CC  & Coordin,Conjunction & and,but,or    & SYM & Symbol              & +,\%,\&\\
  CD  & Cardinal number     & one,two,three & TO  & 'to'                & to\\
  DT  & Determiner          & a,the         & UH  & Interjection        & ah,oops\\
  EX  & Existential         & there         & VB  & Verb,base form      & eat\\
  ... & ...                 & ...           & ... & ...                 & ...\\
  RBR & Adverb,comparative  & faster        & .   & Sentence-final      & (.!?)\\
  RBS & Adverb,superlative  & fastest       & :   & Mid-sentence punc   & (:;...-)\\
  RP  & Particle            & up,off        &     &                     &\\
  \hline  
\end{tabular}\]
\end{table}

\subsection{Regular Expression Table}
  Regex feature is to capture the relationship between the morphology of a token and it's corresponding tag.
For example, a token end with 's' are mostly likely to be plural noun and a token end with 'ly' are more likely to be an
adverb. You can define your own regular expressions to capture the intrinsic characteristics of the given training data.
\begin {table}[h]
\caption {Regular expression table} \label{tab:title} 
\begin{center}
\begin{tabular}{ll||ll}
  pattern & name             & pattern & name\\
  \hline                        
  $\wedge[A-Z][a-z]+\$$    & InitCapital       & $\wedge[A-Z]+\$$  & isAllCapital \\
  $\wedge.*[0-9]+.*\$$     & containsDigit     & $\wedge.+[.]\$$   & endsWithDot\\
  $\wedge.+[,]\$$          & endsWithComma     & $\wedge.+er\$$    & endsWithEr\\
  $\wedge.+est\$$	   & endsWithEst       & $\wedge.+ed\$$    & endsWithEd\\
  $\wedge.+s\$$	           & endsWithS         & $\wedge.+ing\$$   & endsWithIng\\
  $\wedge.+ly\$$	   & endsWithly        & $\wedge.+-.+\$$   & isDashSeparatedWords\\
  $\wedge.*@.*\$$	   & isEmailId         &           & \\
  \hline  
\end{tabular}
\end{center}
\end{table}

\section{Feature Extraction}
The Feature Extraction module provides functionality for basic text-analysis
tasks such as part-of-speech (POS) tagging and named-entity resolution.
At present, six feature types are implemented.
    \begin{itemize}
    \item Edge Feature: transition feature that encodes the transition feature weight from current label to next label.
    \item Start Feature: fired when the current token is the first token in a sentence.
    \item End Feature: fired when the current token is the last token in a sentence.
    \item Word Feature: fired when the current token is observed in the trained dictionary.
    \item Unknown Feature: fired when the current token is not observed in the trained dictionary for at least certain times.
    \item Regex Feature: fired when the current token can be matched by the regular expression.
    \end{itemize}

Advantages of extracting features using SQL statements:
\begin{itemize}
\item [$\star$] Decoupling the feature extraction and other code.
\item [$\star$] Compared with procedure language, SQL is much more easier to understand.
\item [$\star$] Store all the features in tables which avoids the recomputing the features over iterations.Also boost the performance.
\item [$\star$] SQL is naivelly paralleled.
\end{itemize}

\subsection{Column Names Convention and Table Schema}
\subsubsection{Column Names Convention}
The following column names are commonly used in the tables of the following sections.
\begin{itemize}
\item doc\_id: Unique integer indentifier of a document.
\item start\_pos: Position of the token in the document starting from 0.
\item seg\_text: Text token itself.
\item prev\_label: Label of previous token.
\item label: Label of current token.
\item max\_pos: End positon of the document.
\item weight: Feature weight associated with certain feature
\item f\_index: Unique integer identifier of a feature
\item f\_name: Feature name. 
\end{itemize}

\subsubsection{Training and Testing Data Schema}
The text data has to been tokenized before it can be stored in the database. One of the commonly used tokenization program for part-of-speech 
tagging is the Treebank tokenization script.
The following table dipicts how the training/testing data is stored in the database table. 

\begin {table}[h!]
\caption {Training or Testing data} \label{tab:trainingdata} 
\begin{center}
    \footnotesize\tt
\begin{tabular}{lllll||lllll}
  start\_pos & doc\_id & seg\_text & label & max\_pos & start\_pos & doc\_id & seg\_text & label & max\_pos\\
  \hline                        
        0&1&'confidence'&11&36& 1&1&'in'&5&36\\ 
        2&1&'the'&2&36& 3&1&'pound'&11&36\\
        4&1&'is'&31&36& 5&1&'widely'&19&36\\ 
        6&1&'expected'&29&36& 7&1&'to'&24&36\\
        8&1&'take'&26&36& 9&1&'another'&2&36\\
        10&1&'sharp'&6&36& 11&1&'dive'&11&36\\
        12&1&'if'&5&36& 13&1&'trade'&11&36\\
        14&1&'figures'&12&36& 15&1&'for'&5&36\\
        16&1&'september'&13&36& 17&1&','&42&36\\ 
        18&1&'due'&6&36& 19&1&'for'&5&36\\\hline  
\end{tabular}
\end{center}
\end{table}

\subsection{Design Challanges and Work-arounds}
As far as I know, the MADlib C++ abstraction layer doesn't support array of self-defined composite data types or multi-dimentional arrays.
But we do have the need of these complex data types in the implemenation of this module. For example, the 
$viterbi\_mtbl$ table is indeed a two dimentional arrays. Due to the limitations of current C++ abstraction layer, we have to convert the matrix 
to an array and later index the data with  $M[i*n+j]$ instead of the normal way $M[i][j]$. Another example is the data types to represent the 
features. A single feature cannot be represented by a single DOUBLE varialbe but by a unit of struct : $[prev\_label, label, f\_index, start\_pos, exist]$  
But there is no arrays of struct type, we had to represent it with one dimentional array. 
Also we have to store the features for a document using  array of doubles instead of array of struct. 

\subsection{Training Data Feature Extraction}
Given training data as described in the following table, SQLs are writen to extract all the features. It happened that any type of features mentioned above can be extracted out by one single SQL clause which makes the code succinct. We illustrate the training data feature extraction by SQL clause examples.
\paragram{Sample Feature Extraction SQLs}
\begin{itemize}
\item $SQL_1$: Extracting edge features:\\
              \begin{lstlisting}[language=SQL,gobble=4]
                        SELECT doc2.start_pos, doc2.doc_id, 'E.', ARRAY[doc1.label, doc2.label]
                        FROM   segmenttbl doc1, segmenttbl doc2
                        WHERE  doc1.doc_id = doc2.doc_id AND doc1.start_pos+1 = doc2.start_pos
              \end{lstlisting}

\item $SQL_2$: Extracting Regex features:\\
              \begin{lstlisting}[language=SQL,gobble=4]
                        SELECT start_pos, doc_id, 'R_' || name, ARRAY[-1, label]
                        FROM  regextbl, segmenttbl
                        WHERE seg_text ~ pattern
              \end{lstlisting}

%\item $SQL_3$: Extracting start features:\\
%	       \fbox{%
%		       \parbox{0.95\linewidth}{%
%                        SELECT $start\_pos, doc\_id$, $'S.'$, $ARRAY[-1, label]$\\
%                        FROM   segmenttbl\\
%                        WHERE $start\_pos = 0$
%		       }%
%	       }\\
        
%\item $SQL_4$: Extracting end features:\\
%	       \fbox{%
%		       \parbox{0.95\linewidth}{%
%                        SELECT $start\_pos, doc\_id$, $'S.'$, $ARRAY[-1, label]$\\
%                        FROM   segmenttbl\\
%                        WHERE $start\_pos = max\_pos$
%		       }%
%	       }\\
%
%\item $SQL_5$: Extracting word features:\\
%	       \fbox{%
%		       \parbox{0.95\linewidth}{%
%                        SELECT $start\_pos, doc\_id$, $'W\_'||seg\_text$, $ARRAY[-1, label]$\\
%                        FROM   segmenttbl
%		       }%
%	       }\\
%\item $SQL_6$: Unknown features:\\
%	       \fbox{%
%		       \parbox{0.95\linewidth}{%
%                        SELECT $start\_pos, doc\_id$, $U$, $ARRAY[-1, label]$\\
%                        FROM segmenttbl  seg,  dictionary dic\\
%                        WHERE $seg.seg\_text = dic.token$ AND $dic.total <= 1$
%		       }%
%	       }\\

\end{itemize}


The final input table schema which contains all the feature data for the crf learning algorithm is as follows:
\begin{center}
    \begin{tabular}{ | l | l | l | l |}
    \hline
    doc\_id & sparse\_r & dense\_m & sparse\_m \\ 
    \hline
    \end{tabular}
\end{center}

\begin{itemize}

\item 
sparse r feautre(single state feature):(prev\_label, label, f\_index, start\_pos, exist)

\begin{center}
    \begin{tabular}{ | l | l |}
    \hline
    label & Description \\ \hline
    prev\_label & the label of previous token, it is always 0 in r table.\\ 
    label       & the label of the single state feature\\
    f\_index    & the index of the feature in the feature table\\
    start\_pos  & the postion of the token(starting from 0)\\
    exist       & indicate whether the token exist or not in the acutal training data set\\
    \hline
    \end{tabular}
\end{center}

\item 
dense m feature:
(prev\_label, label, f\_index, start\_pos, exist)
\begin{center}
    \begin{tabular}{ | l | l |}
    \hline
    label & Description \\ \hline
    prev\_label & the label of previous token.\\ 
    label       & the label of current token\\
    f\_index    & the index of the feature in the feature table\\
    start\_pos  & the postion of the token in a sentence(starting from 0)\\
    exist       & indicate whether the token exist or not in the acutal training data set\\
    \hline
    \end{tabular}
\end{center}

\item 
sparse m feature:(f\_index, prev\_label, label)
\begin{center}
    \begin{tabular}{ | l | l |}
    \hline
    label & Description \\ \hline
    f\_index    &  the index of the feature in the feature table\\
    prev\_label &  the label of previous token \\
    label       &  the label of current token\\
    \hline
    \end{tabular}
\end{center}

\end{itemize}
For performance consideraton, we split the m feature to dense\_m feature and spare m feature\\
The acutal spare r table is array union of individal r features ordered by the start positon of tokens.
So the function to compute the gradient vector and loglikelihood can scan the feature arrays from begining to end.

\subsection{Learned Model}
The CRF learning algorithm will generate two tables: feature table and dictionary table
Feature table stores all the features and their corresponding feature weight.
The dictionary constains all the tokens and their number of times they appears in the training data.\\
\begin {table}[h!]
\caption {Feature table} \label{tab:title} 
\begin{center}
    \scriptsize\tt
    \begin{tabular}{ | l | l | l | l | l || l | l | l | l | l | }
    \hline
    f\_index & f\_name & prev\_label & label & weight & f\_index & f\_name & prev\_label & label & weight\\
    \hline
    0&'U'&-1&6&2.037& 1&'E.'&2&11&2.746   \\
    2&'W\_exchequer'&-1&13&1.821& 3&'W\_is'&-1&31&1.802 \\
    4&'E.'&11&31&2.469& 5&'W\_in'&-1&5&3.252 \\
    6&'E.'&11&12&1.305& 7&'U'&-1&2&-0.385 \\
    8&'E.'&31&29&1.958& 9&'U'&-1&29&1.422 \\
    10&'R\_endsWithIng'&-1&11&1.061&11&'W\_of'&-1&5&3.652 \\
    12&'S.'&-1&13&1.829& 13&'E.'&24&26&3.282 \\
    14&'W\_helped'&-1&29&1.214& 15&'E.'&11&24&1.556 \\
    \hline
    \end{tabular}
\end{center}
\end {table}

\begin {table}[h!]
\caption {Dictionary table} \label{tab:title} 
\begin{center}
    \scriptsize\tt
    \begin{tabular}{ | l | l || l | l || l | l || l | l | }
    \hline
    token     & total & token   & total & token     & total & token       & total\\
    'freefall'& 1     & 'policy'& 2     & 'measures'&1      & 'commitment'&1\\
    'new'&1& 'speech'&1& '''s'&2& 'reckon'&1\\
    'underlying'&1&'week'&1& 'prevent'&1& 'has'&2\\
    'failure'&1& 'restated'&1&'announce'&1& 'thursday'&1\\
    'but'&1& 'lawson'&1& 'last'&1& 'firm'&1\\
    'exchequer'&1& 'helped'&1& 'sterling'&2& $\ldots$ & $\ldots$\\ 
    \hline
    \end{tabular}
\end{center}
\end {table}


\subsection{Testing Data Feature Extraction}
  This component extracts features from the testing data based on the learned models.
  It will produce two factor tables
  $viterbi\_mtbl$ and  $viterbi\_rtbl$. The $viterbi\_mtbl$
  table and a $viterbi\_rtbl$ table are used to calculate the best label
  sequence for each sentence.
\paragram{Sample Feature Extraction SQLs}
\begin{itemize}
\item $SQL_1$: Extracting unique tokens:\\
              \begin{lstlisting}[language=SQL,gobble=4]
                        INSERT INTO segment_hashtbl
                        SELECT DISTINCT seg_text
                        FROM   segmenttbl
              \end{lstlisting}

\item $SQL_2$: Summerize over all single state features with respect to specific tokens and labels :\\
              \begin{lstlisting}[language=SQL,gobble=4]
                        INSERT INTO viterbi_rtbl
                        SELECT seg_text, label, SUM(value)
                        FROM   rtbl
                        GROUP BY seg_text,label
              \end{lstlisting}
\begin{center}
    \begin{tabular}{ | l | l | l | l |}
    \hline
    doc\_id & viterbi\_mtbl & viterbi\_rtbl \\ 
    \hline
    \end{tabular}
\end{center}

  \begin{itemize}
  \item 
  $viterbi\_mtbl$ table
  encodes the edge features which are solely dependent on upon current label and
  previous y value. The m table has three columns which are prev\_label, label,
  and value respectively.
  If the number of labels in $n$, then the m factor table will $n^2$
  rows. Each row encodes the transition feature weight value from the previous label
  to the current label.
 
  startFeature is considered as a special edge feature which is from the
  beginning to the first token. Likewise, endFeature can be considered
  as a special edge feature which is from the last token to the very end.
  So m table encodes the edgeFeature, startFeature, and endFeature.
  If the total number of labels in the label space is 45 from 0 to 44,
  then the m factor array is as follows:
\begin {table}[h]
\caption {viterbi\_mtbl table} \label{tab:title} 
 
  \[\begin{tabular}{l*{6}{c}r}
   token             & 0   & 1   & 2   & 3   & ... & 43 &  44 \\
   \hline
  -1                 & 2.1 & 1.1 & 1.0 & 1.1 & 1.1 & 2.1 & 1.1  \\
   0                 & 1.1 & 3.9 & 1.2 & 2.1 & 2.8 & 1.8 & 0.8  \\
   1                 & 0.7 & 1.7 & 2.9 & 3.8 & 0.6 & 3.2 & 0.2  \\
   2                 & 0.2 & 3.2 & 3.8 & 2.9 & 0.2 & 0.1 & 0.2  \\
   3                 & 1.2 & 6.9 & 7.8 & 8.0 & 0.1 & 1.9 & 1.7  \\
   ...               & ... & ... & ... & ... & ... & ... & ...  \\
   44                & 8.2 & 1.8 & 3.7 & 2.1 & 7.2 & 1.3 & 7.2  \\
   45                & 1.8 & 7.8 & 5.6 & 9.8 & 2.3 & 9.4 & 1.1  \\
  \end{tabular}\]
\end{table}

  \item 
  $viterbi\_r$ table
  is related to specific tokens. It encodes the single state features,
  e.g., wordFeature, RegexFeature for all tokens. The r table is represented
  in the following way.\\
\begin {table}[h]
\caption {viterbi\_rtbl table} \label{tab:title} 

  \[\begin{tabular}{l*{6}{c}r}
   token             & 0   & 1   & 2   & 3   & ... & 43 &  44 \\
   \hline
   madlib            & 0.2 & 4.1 & 0.0 & 2.1 & 0.1 & 2.5 & 1.2  \\
   is                & 1.3 & 3.0 & 0.2 & 3.1 & 0.8 & 1.9 & 0.9  \\
   an                & 0.9 & 1.1 & 1.9 & 3.8 & 0.7 & 3.8 & 0.7  \\
   open-source       & 0.8 & 0.2 & 1.8 & 2.7 & 0.5 & 0.8 & 0.1  \\
   library           & 1.8 & 1.9 & 1.8 & 8.7 & 0.2 & 1.8 & 1.1  \\
   ...               & ... & ... & ... & ... & ... & ... & ...  \\
  \end{tabular}\]
\end{table}
\end{itemize}


\section{Linear-chain CRF Inference}
The Viterbi algorithm is the popular algorithm to find the top-k most likely labelings of a document for CRF models. 
For the tasks in natural language processing domain, it is sufficient to only generate the best label sequence.  
We chose to use a SQL clause to drive the inference over all documents. 
In Greenplum, Viterbi can be run in parallel over different subsets of the document on a multi-core machine.

\subsection{Parallel CRF Inference}
The $vcrf\_top\_label$ is implemented sequentially and each function call will finish labeling of one document. 
The inference is paralledl in the level of document. We use a SQL statment to drive the inference of all documents.
So, the CRF inference is naivelly parallel. 
\begin{lstlisting}[language=SQL,gobble=4]
        SELECT doc_id, vcrf_top1_label(mfactors.score, rfactors.score)
        FROM   mfactors,rfactors
\end{lstlisting}

\subsection{Viterbi Inference Algorithm}
\[
V(i,y) =
\begin{cases}
 \max_{y^\prime}(V(i-1,y^\prime)) + \textstyle \sum_{k=1}^K \lambda_kf_k(y,y^\prime,x_i), & \text{if } i\ge0 \\
 0, & \text{if } i=-1.
\end{cases}
\]

\subsection{Viterbi Inference output}
The final inference output will produce the the best label sequence for each document and also the conditional probability given the
observed input sequence.
\begin {table}[h]
\caption {Viterbi inference output} 
\begin{center}
    \scriptsize\tt
    \begin{tabular}{ | l | l | l | l | l | }
    \hline
    doc\_id & start\_pos & token & label & probability     \\\hline
    1   & 0    & madlib        & proper noun, singular &0.6\\ 
    1   & 1    & is            & Verb, base form       &0.6 \\
    1   & 2    & an            & determiner            &0.6 \\ 
    1   & 2    & open-source   & adjective             &0.6 \\
    1   & 4    & library       & noun                  &0.6 \\
    1   & 5    & for           & preposition           &0.6 \\
    1   & 6    & scalable      & adjective             &0.6 \\
    1   & 7    & in-dababase   & adverb                &0.6 \\
    1   & 8    & analytics     & noun, singular        &0.6 \\
    1   & 9    & .             & sentence-final punc   &0.6 \\
    2   & 0    & it            & personal pronoun      &0.4 \\ 
    2   & 1    & provides      & verb, base form       &0.4 \\
    2   & 2    & data-parallel & noun                  &0.4 \\
    2   & 2    & $\ldots$      & $\ldots$                    &0.4 \\
    \hline
    \end{tabular}
\end{center}
\end{table}


\printbibliography

\end{document}
